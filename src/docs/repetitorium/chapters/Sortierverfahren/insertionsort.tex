\section{Insertion-Sort}

\textbf{Insertion-Sort} ist ein auf Schlüsselvergleichen basierendes, \textbf{internes}, \textbf{stabiles} Sortierverfahren, das \textbf{intern} sortiert.

\subsection{Methode}

Bei \textbf{Insertion-Sort} wird das zu sortierende Feld in zwei Teilfolgen unterteilt - die linke Teilfolge gilt als sortiert - die rechte Teilfolge gibt die Schlüssel vor, die nun in die linke Teilfolge einsortiert werden sollen.\\
Hierfür wird jeweils ein Schlüssel aus der rechten Teilfolge mit den Schlüsseln in der linken Teilfolge verglichen, und in die Position eingefügt, die der erwarteten Sortierreihenfolge entspricht.

\subsection{Implementierung}

\begin{minted}{java}
    for (int i = 1; i < n; i++) {
        int min = arr[i];
        int j = i;
        while (j > 0 && arr[j - 1] > min) {
            arr[j] = arr[j - 1];
            j--;
        }
        arr[j] = min;
    }
\end{minted}

\subsection{Laufzeit}
Die äußere Schleife wird $n-1$ mal durchlaufen, die innere so oft, bis keine Inversion mehr festgestellt wird\footnote{
    das ist einer der wesentliche Unterschied zu Selection-Sort
}.\\
Im \textbf{worst-case} muss die innere Schleife allerdings in Abhängigkeit von $i$ jeweils $i$-mal durchlaufen werden.\\

Mit

\begin{equation}
    \sum_{i = 1}^{n-1} \sum_{j=1}^i 1
\end{equation}

folgt die Laufzeitkomplexität $O(n^2)$.\\

\begin{tcolorbox}[title={Lineares Laufzeitverhalten}]
    Insertion-Sort eignet  sich sehr gut für \textit{vorsortierte} Felder, wo es ein \textit{lineares Laufzeitverhalten} aufweist (vgl.~\cite[188]{CL22}).
\end{tcolorbox}


