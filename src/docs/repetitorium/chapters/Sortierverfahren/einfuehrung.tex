\section{Grundlagen}

Der Abschnitt wiederholt wesentliches zu \textit{einfachen} Sortierverfahren\footnote{
    vgl. \textit{Güting und Dieker} in~\cite[170]{GD18e}; bei \textit{Ottmann und Widmayer} auch \textit{elementare} Sortierverfahren (vgl.~\cite[82 ff.]{OW17b})
}wie \textbf{Selection-Sort}, \textbf{Insertion-Sort} und \textbf{Bubblesort}, deren Laufzeitkomplexität in $O(n^2)$ liegt\footnote{
im Folgenden bezieht sich $O$ immer auf den \textbf{worst-case}, sofern nicht anders angegeben
}.\\
Außerdem werden \textbf{Divide-and-Conquer}-Verfahren wie \textbf{Merge-Sort} ($O(n\ log\ n)$) und \textbf{Quicksort} ($O(n^2)$) wiederholt.

\subsection{Allgemeine Sortierverfahren}
Unter \textbf{allgemeinen Sortierverfahren} fasst man diejenigen Sortierverfahren zusammen, die Informationen über die relative Anordnung von zu sortierenden Schlüsseln nur über paarweise Schlüsselvergleiche gewinnen (vgl.~\cite[130]{OW17b}).\\
Alle in diesem Abschnitt behandelten Sortierverfahren sind \textit{allgemeine} Sortierverfahren\footnote{
ein Sortierverfahren, dass nicht über Schlüsselvergleiche sortiert, ist \textit{Radixsort} (vgl.~\cite[121]{OW17b})
}.

\subsection{Klassifikation von Sortieralgorithmen}

\textit{Güting und Dieker} klassifizieren Sortieralgorithmen neben der Effizienz und den angewandten Methoden u.a. nach folgenden Kriterien (vgl.~\cite[169 f.]{GD18e}):

\begin{itemize}
    \item \textbf{intern / extern}
    \item[] \textit{interne} Verfahren halten alle Daten im Hauptspeicher, \textit{externe} laden nur einen Teil davon in den Speicher (bspw. beim Sortieren von Daten, die auf Diskette / Platte / Band vorliegen (vgl.~\cite[80]{OW17b})).
    \item \textbf{in place (in situ)}
    \item [] Sortierverfahren, die keinen zusätzlichen Speicherplatz zum Sortieren der Daten benötigen, sortieren \textbfin place.\\
    Die Eingangsfolge und die Ausgangsfolge werden im selben Array dargestellt, Vertauschungen von Schlüsseln erfolgen in genau diesem Array.\\
    Alle in diesem Kapitel behandelten Sortierverfahren fallen in diese Kategorie, bis auf \textbf{Merge-Sort}.
    \item \textbf{stabil}
    \item[] Sortierverfahren, bei denen die Ergebnisfolge gleicher Schlüssel dieselbe ist wie die Ausgangsfolge, nennt man \textbf{stabil}:\blockquote[{\cite[164]{OW17b}}]{
        [...] die Reihenfolge von Elementen mit gleichem Sortierschlüssel wird während des Sortierverfahrens nicht vertauscht.
    }.\\
    Die Sortierverfahren \textbf{Bubblesort}, \textbf{Insertion-Sort} und \textbf{Merge-Sort} sind allesamt stabile Sortierverfahren.\\
    Nicht stabil sind \textbf{Selection-Sort} und \textbf{Quicksort}.
    \item \textbf{natürlich}
    \item[] \textbf{natürliche Sortierverfahren} arbeiten bei vorsortierten Daten schneller als bei unsortierten (bspw. \textbf{Insertion-Sort}, wohingegen \textbf{Selection-Sort} auch bei vorsortierten Daten die gleiche Laufzeitkomplexität aufweist, wie bei unsortierten)
\end{itemize}


\subsection{Untere Schranke}

\blockquote[{\cite[154, Satz 2.4]{OW17b}}]{
    Jedes allgemeine Sortierverfahren benötigt zum Sortieren von $N$ verschiedenen Schlüsseln sowohl im schlechtesten Fall als auch im Mittel wenigstens $\Omega(N\ log\ N)$ Schlüsselvergleiche.
}.


\subsection{divide-and-conquer-Paradigma}

\textit{Güting und Dieker} beschreiben das \textbf{divide-and-conquer}-Paradigma in~\cite[174]{GD18e}:
\begin{itemize}
    \item Ist die Objektmenge klein genug, löse das Problem direkt.
    \item \textit{Divide} Ansonsten zerlege die Objektmenge in mehrere Teilmengen (möglichst gleicher Größe)
    \item \textit{Conquer} Löse das Problem rekursiv für die Teilmenge
    \item \textit{Merge} Berechne aus den für die Teilmengen erhaltenen Lösungen eine Lösung für das Gesamtproblem
\end{itemize}