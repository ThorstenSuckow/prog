\section{Grundlagen}

\subsection{Methodensignatur}
In Java gehört zu der Methodensignatur der Methodenname sowie die formale Parameterliste\footnote {
    Java Language Specification - 8.4.2. Method Signature: \url{https://docs.oracle.com/javase/specs/jls/se21/html/jls-8.html#jls-8.4.2} - abgerufen 07.03.2024
}.

\subsection{Überschreiben}
Eine Methode wird in einer Unterklasse \textit{überschrieben}, wenn Signatur und Rückgabetyp der Methode der Unterklasse mit der
Methode der Oberklasse übereinstimmen\footnote{stimmt nur die Signatur überein, kommt es zu einem Compilerfehler}.\\


\subsection{Überladen}
Eine Methode wird \textit{überladen}, wenn eine Methode mit gleichem Namen bereitgestellt wird, aber eine unterschiedliche Signatur
aufweist.\\
Der Rückgabetyp darf abweichen:
\blockquote[Java Language Specification - 8.4.9. Overloading\footnote{\url{https://docs.oracle.com/javase/specs/jls/se21/html/jls-8.html#jls-8.4.9} - abgerufen 07.03.2024}]{
    There is no required relationship between the return types or between the throws clauses of two methods with the same name, unless their signatures are override-equivalent.
}

\noindent
Unter einem \textit{mehrdeutigen} Methodenaufruf versteht man einen Aufruf zu einer überladenen Methode, die der Compiler nicht eindeutig zuordnen kann\footnote{
s. ``15.12.2.5. Choosing the Most Specific Method``: \url{https://docs.oracle.com/javase/specs/jls/se21/html/jls-15.html#jls-15.12.2.5} - abgerufen 07.03.2024
}.\\
Im Folgenden Beispiel wird die Methode \code{foo(x: int, y: double):void} durch \code{foo(x: double, y: int):void} überladen:

\begin{minted}[mathescape,
    gobble=2]{java}
    public void foo(int x, double y) {
        ...
    }

    public void foo(double x, int y) {
        ...
    }
\end{minted}\\

Den Aufruf

\begin{minted}[mathescape,
    gobble=2]{java}
    foo(1, 2);
\end{minted}\\

\noindent
quittiert der Compiler mit einer Fehlermeldung ``Ambiguous method call.``.\\
An dieser Stelle ist explizites casten nötig, damit für den Compiler klar ist, welche Methode aufgerufen werden soll:

\begin{minted}[mathescape,
    gobble=2]{java}
    foo(1, (double)2);
\end{minted}\\


\section{Vererbung}

\begin{itemize}
    \item Wird in einer Superklasse eine \textit{Objektmethode} definiert, die mit gleicher Signatur in der Unterklasse implementiert wird, dann \textbf{überschreibt} die Methode der Unterklasse die der Oberklasse
    \item Wird in einer Superklasse eine \textit{Klassenmethode} definiert, die mit gleicher Signatur in der Unterklasse implementiert wird (ebenfalls als \code{static} deklariert), dann \textbf{überdeckt} die Methode der Unterklasse die der Oberklasse\footnote{
        s. ``Overriding and Hiding Methods``: \url{https://docs.oracle.com/javase/tutorial/java/IandI/override.html} - abgerufen 12.03.2024
    }.
    \item Statische Methoden / Eigenschaften werden vererbt, können aber in Unterklassen \textit{nicht} überschrieben werden, sondern nur überdeckt
\end{itemize}

\begin{minted}{java}
    class A {

        static int snafu = 42;

        static int baz = 0;

        public void foo() {}

        public static void bar() {
            System.out.println("in A");
        }
    }

    class B extends A {

        /**
         * überdeckt A.baz
         */
        static int baz = 2;

        /**
         * überschreibt foo()
         */
        public void foo() {

            /**
             * Ruf implizit A.snafu auf
             */
            System.out.println(snafu);

        }

        /**
         * überdeckt bar()
         */
        public static void bar() {
            System.out.println("in A");
        }
    }
\end{minted}
