\section{Zugriffsmodifizierer}


Tabelle~\ref{tab:modifier} listet die verfügbaren Zugriffsmodifizierer in Java auf\footnote{
    ``The Java™ Tutorials - Controlling Access to Members of a Class``: \url{https://docs.oracle.com/javase/tutorial/java/javaOO/accesscontrol.html} - abgerufen 08.03.2024
}

\setlength{\tabcolsep}{0.5em}
\renewcommand{\arraystretch}{1.5}%
\begin{table} %[hbtp]
    \centering
    \begin{tabular}{|c | c | c | c | c| c|}
        \hline
        \textbf{Modifizierer} & Klasse & Package & Unterklasse & Welt & UML \\
        \hline
        \code{public}   & $\checkmark$   & $\checkmark$ & $\checkmark$ & $\checkmark$  & \code{+} \\
        \hline
        \code{protected}   & $\checkmark$   & $\checkmark$ & $\checkmark$ & $-$  & \code{#} \\
        \hline
        kein Modifizierer   & $\checkmark$   & $\checkmark$ & $-$ & $-$  & \code{~} \\
        \hline
        \code{private}   & $\checkmark$   & $-$ & $-$ & $-$ & \code{-}  \\
        \hline
    \end{tabular}
    \caption{Zugriffsmodifizierer und ihre Sichtbarkeiten. Die UML-Notation bzgl. der Sichtbarkeiten wurde im Skript (Teil 1), Abschnitt 5.4.3 und 5.5 \textit{nicht} verwendet und ist der Vollständigkeit halber aufgelistet (s. a.~\cite[404, Tabelle 6.4]{Ull23} sowie \cite[251 ff.]{Oes05}). }
    \label{tab:modifier}
\end{table}

\noindent
Wenn kein expliziter Zugriffsmodifizierer angegeben wurde, ist die Sichtbarkeit implizit \textit{package-private} (Zeile 3 in Tabelle~\ref{tab:modifier}).\\

\noindent
Die Sichtbarkeit einer überschrieben Methode darf nicht reduziert werden.
Erweitern ist dagegen möglich:

\begin{minted}[mathescape,
    gobble=2]{java}
    class A {
        protected function foo() {
        }

        public function bar() {
        }
    }

    class B extends A {
        public function foo() { // Sichtbarkeit von foo()
        }                       // erweitert

        protected function bar() { // Compilerfehler:
        }                          // "Attempting to assign weaker
                                   // access privileges"

    }

\end{minted}


\section{Packages}
Das Package-Konzept in Java repräsentiert kein hierarchisches Modell:\blockquote{
    [...], packages appear to be hierarchical, but they are not.\footnote{
        ``The Java™ Tutorials - Apparent Hierarchies of Packages``: \url{https://docs.oracle.com/javase/tutorial/java/package/usepkgs.html} - abgerifen 08.03.2024
    }
}\\
Aus diesem Grund kennt Java so etwas wie ``innere Pakete`` nicht.\\

\noindent
Verdeutlicht wird dies anhand des Beispiels von \code{import}-Deklarationen, die Wildcards (``*``) in Verbindung mit einem qualifizierten Namen nutzen\footnote{
    Eine sog. ``Type-Import-on-Demand Declaration``. Siehe  ``Java Language Specification - 7.5.2. Type-Import-on-Demand Declarations``: \url{https://docs.oracle.com/javase/specs/jls/se21/html/jls-7.html#jls-7.5.2} - abgerufen 08.03.2024
}.\\
So können alle öffentlichen Klassen aus dem Paket \code{java.awt} in Typdeklarationen einer Übersetzungseinheit\footnote{
    ``Compilation Unit``. Siehe hierzu ``Java Language Specification - 7.3. Compilation Units``: \url{https://docs.oracle.com/javase/specs/jls/se8/html/jls-7.html#jls-7.3} - abgerufen 08.03.2024
} über ihren Namen referenziert werden, wenn wie folgt importiert wird:

\begin{minted}[mathescape,
    gobble=2]{java}
    import java.awt.*;
\end{minted}

\noindent
Dies schließt allerdings \unerline{nicht} die Klassen ein, die in Paketen liegen, die ihrem qualifizierten Namen nach scheinbar ``in`` \code{java.awt} liegen,
wie bspw. \code{java.awt.color}.
Diese müssen separat importiert werden:

\begin{minted}[mathescape,
    gobble=2]{java}
    import java.awt.*;
    import java.awt.color.*;
\end{minted}