\section{Objektinitialisierung}

Die Sprachspezifikationen beschreiben die Objekterzeugung und die Reihenfolge der damit verbundenen Konstruktor-Aufrufe und Attribut-Initialiserung vereinfacht wie folgt\footnote{
s. ``12.5. Creation of New Class Instances``: \url{https://docs.oracle.com/javase/specs/jls/se21/html/jls-12.html#jls-12.5} - abgerufen 13.03.2024
}:\\

\noindent
Führe \textit{rekursiv} aus:\\

\begin{enumerate}
    \item Der Konstruktor wird mit den Argumenten aufgerufen
    \item Ruft der Konstruktor über \code{this} einen anderen Konstruktor auf, gehe zu Schritt 1
    \item Ruft der Konstruktor implizit oder explizit einen Konstruktor der Elternklasse auf, gehe zu Schritt 1
    \item Die Objektattribute werden initialisiert
    \item Die übrigen Anweisungen innerhalb des Konstruktors werden ausgeführt
\end{enumerate}

\noindent
Im Folgenden ein Beispiel für die Initialisierungsreihenfolge durch den Aufruf \code{new C()}.\\

\begin{minted}{java}
    class Dummy {
        public Dummy(String from) {
            System.out.println("Dummy for " + from);
        }
    }

    class A {
        Dummy d = new Dummy("A");
        public A() {
            System.out.println("A");
        }
    }

    class B extends A {
        public B() {
            System.out.println("B");
        }
    }

    class C extends B {
        Dummy d = new Dummy("C");
        public C() {
            System.out.println("C");
        }
    }
\end{minted}\\

\noindent
Die Ausgabe lautet:


\begin{minted}{text}
    Dummy for A
    A
    B
    Dummy for C
    C
\end{minted}