\section{Konstruktoren}

Ein \textbf{allgemeiner Konstruktor} ist ein Konstruktor mit einer formalen Parameterliste (vgl.~\cite[424]{Ull23}).\\

\noindent
Der \textbf{Standard-Konstruktor} ist ein \textit{parameterloser} Konstruktor (vgl.~\cite[423]{Ull23}). \\

\noindent
Konstruktoren in Java sind nicht mit Objektmethoden gleichzusetzen, und zählen \textit{nicht} zu den vererbbaren Eigenschaften einer Klasse.\\

\noindent
Wenn für eine Klasse kein Konstruktor definiert wurde, stellt der Compiler einen \textbf{Standard-Konstruktor}
zur Verfügung\footnote{
    The Java Language Specification - 8.8.9. Default Constructor: \url{https://docs.oracle.com/javase/specs/jls/se21/html/jls-8.html#jls-8.8.9}↩
}.


\begin{minted}[mathescape,
    gobble=2]{java}
    class A {

        // implizit deklarierter default constructor
        // public A() {
        // }

        public static void main(String[] args) {
            A a = new A(); // ruft den implizit deklarierten Standard-Konstruktor
                           // auf
        }

    }
\end{minted}

\noindent
Wenn ein \textit{explizit} deklarierter Konstruktor vorhanden ist, stellt der Compiler keinen Standard-Konstruktor zur Verfügung.\\

\noindent
Im folgenden Beispiel führt \code{new A()} zu einem Compiler-Fehler, da kein parameterloser Konstruktor explizit oder implizit für \code{A} deklariert ist.

\begin{minted}[mathescape,
    gobble=2]{java}
    class A {

        int a;

        // explizit deklarierter Konstruktor
        // unterbindet die Bereitstellung eines implizit
        // deklarierten Standard-Konstruktor
        public A(int value) {
            a = value;
        }

        public static void main(String[] args) {
            A a1 = new A(i); // ruft den explizit deklarierten Konstruktor auf
            A a2 = new A();  // fuehrt zu einem Compiler-Fehler
        }

    }
\end{minted}

\noindent
Wenn ein Konstruktor nicht \textit{explizit} einen Konstruktor seiner Elternklasse aufruft, ruft er \textit{implizit} \code{super()} auf.

\noindent
Wenn die Elternklasse in solchen Fällen keinen parameterlosen Konstruktor deklariert hat, wird ein Compiler-Fehler erzeugt.

\noindent
Im folgenden Beispiel erweitert \code{B} die Klasse \code{A}. \code{B} hat einen Konstruktor deklariert, dessen
formale Parameterliste \code{int} ist. Dieser Konstruktor ruft implizit \code{super()} auf:

\begin{minted}[mathescape,
    gobble=2]{java}
    class A {

    }

    class B {

        int val;

        public B (int a) {
            // implizit aufgerufen:
            // super();
            val = a;
        }

        public static void main(String[] args) {
            B b = new B(1);
        }

    }
\end{minted}

\noindent
Ergänzend sei daran erinnert, dass jede Klasse in Java direkt oder indirekt von \code{java.lang.Object} erbt\footnote{
    The Java Tutorials - Object as a Superclass: \url{https://docs.oracle.com/javase/tutorial/java/IandI/objectclass.html}↩
}: Der Konstruktor der Klasse \code{java.lang.Object} ist also in einer Konstruktor-Aufrufkette enthalten.\\

\noindent
Im nächsten Beispiel erbt \code{C}  von \code{B}. Ein Standard-Konstruktor wird implizit für \code{C} deklariert. Allerdings
führt der Versuch, ein Objekt vom Typ \code{C} zu erzeugen, zu einem Compiler-Fehler, da \code{B} keinen parameterlosen
Konstruktor - den Standardkonstruktor - deklariert:


\begin{minted}[mathescape,
    gobble=2]{java}
    class C extends B {
        public static void main(String[] args) {
            C c = new C(); // Compiler-Fehler
        }
    }
\end{minted}

\noindent
Damit ein Objekt vom Typ \code{C} erzeugt werden kann, benötigt \code{C} einen Standardkonstruktor mit einem expliziten
Aufruf des Konstruktors von \code{B} - da \code{B} selber keinen Standardkonstruktor deklariert hat, muss der Konstruktor
mit der Signatur \code{B(int)} aufgerufen werden:

\begin{minted}[mathescape,
    gobble=2]{java}
    class C extends B {

        public C() {
           super(2); // ruft den Konstruktor von B mit dem Argument 2 auf
        }

        public static void main(String[] args) {
            C c = new C();
        }

    }
\end{minted}

\noindent
Das zeigt, dass in Java bei einem explizit deklarierten Konstruktor als \textit{erstes}
der Konstruktor der Elternklasse aufgerufen wird (implizit durch den Compiler) bzw. aufgerufen werden muss (falls explizit implementiert)\footnote{
    es kann auch zunächst durch \textit{this()} ein anderer Konstruktor aufgerufen werden; s.a.
    ``The Java Language Specification - 8.8.7. Constructor Body``: \url{https://docs.oracle.com/javase/specs/jls/se21/html/jls-8.html#jls-8.8.7} - abgerufen 12.03.2024
}.\\

\noindent
Es folgen ergänzende Beispiele.\\
Details sind den Quelltextkommentaren zu entnehmen.

\begin{minted}[mathescape,
    gobble=2]{java}
    class A {
        // impliziter Standard-Konstruktor: vorhanden

        // impliziter Standard-Konstruktor: ruft Konstruktor von java.lang.Object
        // auf
    }


    class B extends A {

        // impliziter Standard-Konstruktor: vorhanden

        // impliziter Standard-Konstruktor: ruft Konstruktor von A auf
    }


    class C {

        // impliziter Standard-Konstruktor: nicht vorhanden

        public C() {
            // impliziter Aufruf von java.lang.Object's Konstruktor
            System.out.println("C created.");
        }
    }


    class D extends C {

        // impliziter Standard-Konstruktor: vorhanden

        // impliziter Standard-Konstruktor: ruft Konstruktor von C auf

    }


    class E extends D {

        // impliziter Standard-Konstruktor: nicht vorhanden

        public E() {
            // impliziter Aufruf von D's Konstruktor
            System.out.println("E created.");
        }
    }


    class F extends E {

        // impliziter Standard-Konstruktor: nicht vorhanden

        public F(int x) {
            // impliziter Aufruf von E's Standardkonstruktor
            System.out.println("F created");
        }
    }


    class G extends F {

        // impliziter Standard-Konstruktor: nicht vorhanden

        public G(int x) {
            super(x); // expliziter Aufruf von F's Konstruktor

            // Das Auskommentieren der o.a. Anweisung fuehrt zu einem impliziten
            // Aufruf von 'super()', was einen Compiler-Fehler produziert:
            // Da kein Standardkonstruktor in F deklariert ist, muss dem
            // Konstruktor explizit mitgeteilt werden, welcher Konstruktor der
            // Elternklasse aufgerufen werden soll.
            System.out.println("G created");
        }
    }
\end{minted}