\section{Schleifen}

\subsection{for-Schleife}

Eine \code{for}-Schleife ist folgendermaßen aufgebaut\footnote{
s. ``14.14.1. The basic for Statement``: \url{https://docs.oracle.com/javase/specs/jls/se21/html/jls-14.html#jls-14.14.1} - abgerufen 12.03.2024
}:\\

\noindent
\code{for} ([\code{ForInit}] ; [\code{Expression}] ; [\code{ForUpdate}]) \code{Statement}\\

\begin{itemize}
    \item \code{ForInit}, \code{Expression}, \code{ForUpdate} sind optional.
    \item ist \code{Expression} angegeben, \textbf{muss} der Ausdruck ein \textit{boolescher} Ausdruck sein
    \item \code{ForInit} darf eine kommaseparierte Liste von lokalen Variablendeklarationen sein, oder eine kommaseparierte Liste von Ausdrucksanweisungen
    \item \code{ForUpdate} darf eine kommaseparierte Liste von Ausdrucksanweisungen sein
    \item \code{ForUpdate} wird \textit{nach} einem Durchlauf der Schleife ausgeführt, sofern der boolesche Ausdruck (falls vorhanden) zu \code{true} ausgewertet wurde.
    Die Reihenfolge entspricht grob:
    \begin{enumerate}
        \item Ausführen von \code{ForInit}
        \item Auswertung der \code{Expression}
        \item Ausführen des \code{Statement}s
        \item Ausführen von \code{ForUpdate}
    \end{enumerate}

\end{itemize}\\

\noindent
Beispiele für gültige \code{for}-Schleifen:

\begin{minted}{java}
    // endlosschleife, boolescher Ausdruck nicht angegeben
    // i und a sind nur in dem Block der for-Schleife sichtbar
    for (int i = 0, a = 1; ; i++) {
        System.out.println(a++);
    }

    int z = 0;
    int a = 0;
    for (System.out.println("Hello World!"),
         System.out.println("starte Schleife");
        z < 10;
        z++,
        System.out.println("z und a jetzt bei: " + z + "(z), " + a + "(a)")) {
        System.out.println("im Schleifenrumpf: " + z + "(z), " + a + "(a)");
        a++;
    }


\end{minted}

