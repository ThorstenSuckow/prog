\section{Typumwandlung}


Numerische Datentypen werden in Java \textit{implizit} oder \textit{explitzit} konvertiert.\\

\noindent
Zu den numerischen Datentypen\footnote{
s. ``Chapter 4. Types, Values, and Variables``: \url{https://docs.oracle.com/javase/specs/jls/se21/html/jls-4.html#jls-4.2.5} - abgerufen 07.03.2024
} gehören

\begin{itemize}
    \item \code{byte}
    \item \code{short}
    \item \code{int}
    \item \code{long}
    \item \code{char}
    \item \code{float}
    \item \code{double}
\end{itemize}\\



\noindent
Eine \textit{implizite} Konvertierung erfolgt, wenn eine Variable eines Datentyps mit einer niedrigeren Präzision einer Variablen eines Datentyps einer höheren Präzision zugewiesen wird:

\begin{minted}[mathescape,
    gobble=2]{java}
    int i = 32;
    long l = i;

    double d = 32.0f;

    short s = 'd';
\end{minted}\\

\noindent
Eine \textit{explizite} Konvertierung muss erfolgen, wenn eine Zuweisung von einem Datentyp mit einer höheren Auflösung einem Datentyp mit niedriger Ausflösung erfolgen soll.\\
Hierbei muss explizit \textbf{gecasted} werden.

\begin{minted}[mathescape,
    gobble=2]{java}
    int i = (int)32L;

    float f = (float)32.0;

    short s = -65_536;
    char c = (char)s;
\end{minted}\\

\noindent
Grundsätzlich können alle Datentypen, die Zahlen repräsentieren, untereinander zugewiesen werden.\\
Hierzu zählt auch der Datentyp \code{char}, der einen numerischen Wertebereich von $[0, 65.535]$ ($2^{16}-1$) abdeckt.

\noindent
Eine Konvertierung zwischen \code{boolean} und nuemrischen Werten ist weder mit explizitem noch implizitem casten möglich\footnote{
``A boolean value may be cast to type boolean, Boolean, or Object [...]. No other casts on type boolean are allowed.`` (s. \url{https://docs.oracle.com/javase/specs/jls/se21/html/jls-4.html#jls-4.2.5} - abgerufen 07.02.2024)
}; allerdings kann das Ergebnis eines Ausdrucks als boolescher Wert gespeichert werden:
\begin{minted}[mathescape,
    gobble=2]{java}
    int i = 1;
    boolean b = i!=0;
\end{minted}\\

\subsection{Auto-Promotion bei arithmetischen Operationen}

Bei arithmetische Operationen (\code{+}, \code{-}, \ldots) wird der Datentyp der Operanden auf den Datentyp mit der höheren Präzision erweitert.\\

\noindent
Im folgenden Beispiel soll ein \code{short}- zu einem \code{int}-Wert addiert werden.
Der Compiler lehnt die Anweisung mit dem Hinweis ab, dass der Ausdruck \code{(short)3 + i} vom Typ \code{int} ist.
Für eine erfolgreiche Zuweisung müßte das Ergebnis erst zu \code{short} gecasted werden.

\begin{minted}[mathescape,
    gobble=2]{java}
    int i = 4;
    short s = (short)3 + i;
\end{minted}\\

Dagegen funktioniert folgendes Beispiel ohne weitere, da der Compiler den Ausdruck während der Kompilierung berechnet\footnote{
    ``Some expressions have a value that can be determined at compile time. These are constant expressions [...].`` (s. \url{https://docs.oracle.com/javase/specs/jls/se21/html/jls-15.html#jls-15.2} - abgerufen 07.03.2024)
}, keinen Überlauf feststellt und das Ergebnis des Ausdrucks sicher dem \code{short}-Typ zuweisen kann:

\begin{minted}[mathescape,
    gobble=2]{java}
    short s = 3 + 4;
\end{minted}

Bei folgendem Code hingegen stellt der Compiler einen Überlauf fest und lehnt die Zuweisung ab\footnote{Wertebereich \textit{short}: $[-2^{15}, 2^{15} -1]$}:

\begin{minted}[mathescape,
    gobble=2]{java}
    short s = 1 + 32767;
\end{minted}\\

\noindent
Die Tabelle ~\ref{tab:typeconvert} fasst die Zuweisungskompatibilität einfacher Datentypen zusammen, bei denen kein explizites Casting erforderlich ist, eine kompaktere Übersicht findet sich in Tabelle~\ref{tab:typcompact}.\\
Für weitere Informationen sei auf die Sprachspezifikationen, Abschnitt ``5.1. Kinds of Conversion``\footnote{
    \url{https://docs.oracle.com/javase/specs/jls/se21/html/jls-5.html#jls-5.1} - abgerufen 07.03.2024
}, verwiesen.


{\renewcommand{\arraystretch}{1.5}%
\setlength{\tabcolsep}{12pt}%
    \begin{table} %[hbtp]
        \begin{center}
            \begin{tabular}{|c | c | c |c |c | c | c| c|}
                \hline
                        &  \textbf{byte} &\textbf{short} & \textbf{char}        &\textbf{int} &\textbf{long} &\textbf{float} & \textbf{double} &
                \hline
                \textbf{byte}     & \checkmark &      -               &     -                      &       -       &       -       &       -       &       -     \\
                \hline
                \textbf{short}    & \checkmark & \checkmark           &     -                       &       -       &       -       &       -       &       -     \\
                \hline
                \textbf{char}  &      -     &      -                &     \checkmark                    &       -       &       -       &       -       &       -     \\

                \hline
                \textbf{int}  & \checkmark & \checkmark             &     \checkmark                  &  \checkmark  &       -       &       -       &       -     \\
                \hline
                \textbf{long}  & \checkmark & \checkmark              &     \checkmark                    &  \checkmark  &  \checkmark &       -       &       -     \\
                \hline
                \textbf{float} & \checkmark & \checkmark             &     \checkmark                  &  \checkmark  &  \checkmark  &        \checkmark       &       -     \\
                \hline
                \textbf{double}& \checkmark & \checkmark            &     \checkmark                   &  \checkmark  &  \checkmark  &        \checkmark       &      \checkmark    \\
                \hline
            \end{tabular}
            \caption{Zuweisungskompatibilität einfacher Datentypen.
            Ein \checkmark bedeutet, dass kein explizites casten von dem Datentyp (horizontale Spalte) zu dem Datentyp (vertikale Spalte) erforderlich ist.}
            \label{tab:typeconvert}
        \end{center}
    \end{table}}


    {\renewcommand{\arraystretch}{1.5}%
\setlength{\tabcolsep}{12pt}%
\begin{table} %[hbtp]
    \begin{center}
        \begin{tabular}{|l | l |}
            \hline
            \textbf{von}  & \textbf{nach} &
            \hline
            byte & short, int, long, float, double \\
            \hline
            short & int, long, float, double \\
            \hline
            char & int, long, float, double \\
            \hline
            int & long, float, double \\
            \hline
            long & float, double \\
            \hline
            float & double \\
            \hline
        \end{tabular}
        \caption{Implizite Typumwandlung in Java in der kompakten Übersicht. (Quelle: in Anlehnung an~\cite[145, Tabelle 2.12]{Ull23})}
        \label{tab:typcompact}
    \end{center}
\end{table}}
