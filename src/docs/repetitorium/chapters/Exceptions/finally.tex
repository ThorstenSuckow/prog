\section{try... finally}

Bei Exceptions, die nicht abgefangen werden, sorgt \code{finally} für ein Ausführen des durch \textit{finally} eingeleiteten Anweisungsblock auch im Fall einer Exception\footnote{
    Java Language Specification - 14.20.2. Execution of try-finally and try-catch-finally : \url{https://docs.oracle.com/javase/specs/jls/se21/html/jls-14.html#jls-14.20.2} - abgerufen 26.01.2024
}, wie folgendes Beispiel zeigt:

\begin{minted}{java}
    public class TryCatchDemo {

        public String  m1(boolean exc) {
            try {
                System.out.println("try 1.");
                if (exc) {
                    throw new Exception();
                }
                System.out.println("try 2.");
                return "foo";
            } catch (Exception e) {
                System.out.println("Exception.");
            } finally {
                System.out.println("finally.");
            }

            return "return m1.";
        }
        public static void main(String[] args) {
            TryCatchDemo demo = new TryCatchDemo();
            System.out.println(demo.m1(false));
            System.out.println();
            System.out.println(demo.m1(true));
        }
    }
\end{minted}\\

\noindent
Die Ausgabe des Programms lautet:


\noindent
\begin{minted}[mathescape,
    numbersep=5pt,
    gobble=2,
    frame=none,
    framesep=2mm]{bash}
    try 1.
    try 2.
    finally.
    foo

    try 1.
    Exception.
    finally.
    return m1.
\end{minted}\\