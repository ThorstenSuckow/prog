\section{Algorithmus}

Die \textbf{Effizienz} eines Algorithmus wird im wesentlichen durch seine \textbf{Laufzeit} und seinen \textbf{Speicherplatzverbrauch} bestimmt.\\

\noindent
Bei der Betrachtung kleinerer Problemgrößen fällt bei der Wahl eines Algorithmus weniger seine Effizienz ins Gewicht, sondern eher die Einfachheit seiner Implementierung und seine Verständlichkeit (vgl. \cite[5 f.]{GD18a}).\\
So läuft bspw. ein Implementierung von Insertion-Sort ($O(n^2)$), die zur Sortierung $8*n^2$ Operationen benötigt, für Eingabemengen $n \leq 43$ schneller sortieren als eine Implementierung von Merge-Sort ($O(n\ log(n))$), die $64 * (n\ log(n))$ Schritte für die Sortierung benötigt.