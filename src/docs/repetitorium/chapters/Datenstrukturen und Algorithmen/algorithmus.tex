\section{Algorithmus}

Die \textbf{Effizienz} eines Algorithmus wird im wesentlichen durch seine \textbf{Laufzeit} und seinen \textbf{Speicherplatzverbrauch} bestimmt.\\

\noindent
Zu den weiteren Eigenschaften eines Algprithmus gehören

\begin{itemize}
    \item
\end{itemize}

\noindent
Bei der Betrachtung kleinerer Problemgrößen fällt bei der Wahl eines Algorithmus weniger seine Effizienz ins Gewicht, sondern eher die Einfachheit seiner Implementierung und seine Verständlichkeit (vgl. \cite[5 f.]{GD18a}).\\
So läuft bspw. ein Implementierung von Insertion-Sort ($O(n^2)$), die zur Sortierung $8*n^2$ Operationen benötigt, für Eingabemengen $n \leq 43$ schneller sortieren als eine Implementierung von Merge-Sort ($O(n\ log(n))$), die $64 * (n\ log(n))$ Schritte für die Sortierung benötigt.


\subsection{Elementaroperationen}

Unter \textbf{Elementaroperationen} versteht man u.a. folgende Operationen (vgl.~\cite[6]{GD18a}):

\begin{itemize}
    \item Zuweisungen: \code{int b = 3}
    \item Vergleiche: \code{if (b <= a)...}
    \item arithmetische Operationen: \code{b+3}
    \item Arrayzugriffe: \code{feld[j]}
\end{itemize}

Die \textbf{Kosten} für eine einzelne Elementaroperation belaufen sich auf $1$ - die Kosten eines Befehls ergeben sich aus der Summe aller für diesen Befehl durchgeführten Elementaroperationen:


\begin{itemize}
    \item \code{int z = b + c;}
    \item[] $1$ Zuweisung, $1$ arithmetische Operation $\rightarrow$ Kosten: $2$
    \item \code{int z = b += feld[j];}
    \item[] $2$ Zuweisungen, $1$ arithmetische Operation, $1$ indizierter Zugriff $\rightarrow$ Kosten: $4$
    \item arithmetische Operationen \code{b+3}
    \item \code{goto 10;}
    \item[] keine Elementaroperation vorhanden $\rightarrow$ Kosten: $0$
\end{itemize}


