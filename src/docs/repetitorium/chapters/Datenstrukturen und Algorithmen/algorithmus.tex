\section{Algorithmus}

Die \textbf{Effizienz} eines Algorithmus wird im wesentlichen durch seine \textbf{Laufzeit} und seinen \textbf{Speicherplatzverbrauch} bestimmt.\\



\noindent
Bei der Betrachtung kleinerer Problemgrößen fällt bei der Wahl eines Algorithmus weniger seine Effizienz ins Gewicht, sondern eher die Einfachheit seiner Implementierung und seine Verständlichkeit (vgl. \cite[5 f.]{GD18a}).\\
So läuft bspw. eine Implementierung von Insertion-Sort ($O(n^2)$), die zur Sortierung $8*n^2$ Operationen benötigt, für Eingabemengen $n \leq 43$ schneller als eine Implementierung von Merge-Sort ($O(n\ log(n))$), die $64 * (n\ log(n))$ Schritte für die Sortierung benötigt\footnote{
Diesbzgl. hatte der Selbsttest ``D+A-Selbsttest-02: O-Notation`` die Frage gestellt, in welchem Fall die Beurteilung eines Algorithmus bezüglich der Komplexitätsklasse nicht unbedingt angemessen sei. Die richtige Antwort hierzu lautete: ``Bei sehr kleinen Eingabemengen``.
}.\\

\begin{tcolorbox}[title={Optimaler Algorithmus}]
    ``Ein Algorithmus heißt (asymptotisch) \textbf{optimal}, wenn die obere Schranke für seine Laufzeit mit der unteren Schranke für die Komplexität des Problems zusammenfällt`` (~\cite[20]{GD18a}).\\

    \noindent
    Sortieralgorithmen mit einer Laufzeit von $O(n\ log\ n)$ sind \textbf{optimal}, bspw. \textbf{Merge-Sort}.
\end{tcolorbox}



\subsection{Elementaroperationen}

Unter \textbf{Elementaroperationen} versteht man u.a. folgende Operationen (vgl.~\cite[6]{GD18a}):

\begin{itemize}
    \item Zuweisungen: \code{int b = 3}
    \item Vergleiche: \code{if (b <= a)...}
    \item arithmetische Operationen: \code{b+3}
    \item Arrayzugriffe: \code{feld[j]}
\end{itemize}

\noindent
Die \textbf{Kosten} für eine einzelne Elementaroperation belaufen sich auf $1$ - die Kosten eines Befehls ergeben sich aus der Summe aller für diesen Befehl durchgeführten Elementaroperationen:


\begin{itemize}
    \item \code{int z = b + c;}
    \item[] $1$ Zuweisung, $1$ arithmetische Operation $\rightarrow$ Kosten: $2$
    \item \code{int z = b += feld[j];}
    \item[] $2$ Zuweisungen, $1$ arithmetische Operation, $1$ indizierter Zugriff $\rightarrow$ Kosten: $4$
    \item \code{goto 10;}
    \item[] keine Elementaroperation vorhanden $\rightarrow$ Kosten: $0$
\end{itemize}


