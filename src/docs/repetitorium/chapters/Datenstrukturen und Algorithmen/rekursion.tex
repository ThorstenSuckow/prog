\section{Rekursion}

Die \textbf{Rekursionstiefe} entspricht der \textbf{Höhe} des Aufrufbaumes - der erste Aufruf der Funktion wird i.d.R. nicht zu der Rekursionstiefe gezählt.\\

\noindent
Anzahl gleichzeitig aktiver \textit{rekursiver} Aufrufe; der erste Aufruf wird i.d.R. nicht dazugezählt und entspricht der ``Wurzel`` des Rekursionsbaums (vgl. Skript (Teil 2) S.34)\footnote{
bzgl. der Begriffsbestimmung siehe hierzu auch \cite[144 f.]{CK75}.
}.
Die Rekursionstiefe entspricht damit der \textit{Höhe} des Rekursionsbaumes.