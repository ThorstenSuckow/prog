\section{Hashing}

Bei allen dynamischen Anwendungen sollte \textbf{offenes Hashing} verwendet werden.\\
\textbf{Geschlossenes Hashing} sollte nur dann verwendet werden, wenn die Gesamtzahl der Elemente von vornherein beschränkt ist, keine oder nur wenige Elemente entfernt werden müssen und ein Belegungsgrad von 60\% - 70\% nicht überschritten wird\footnote{s. Skript (Teil 2) S.141 sowie \cite[126]{GD18d}}.

\subsection{Quadratisches Sondieren}

\blockquote[{\cite[207]{OW17d}}]{
Wenn $m$ eine Primzahl der Form $4i+3$ ist, dann ist garantiert, dass die Sondierungsfolge eine Permutation der Hashadressen 0 bis $m-1$ ist [...]
}

\subsection{Löschen von Schlüsseln bei geschlossenem Hashing}
Wird ein Schlüssel $k$ gelöscht, wird der Bucket für ein einzufügendes Element als \textit{frei} betrachtet, bei der Suche nach einem Element als \textit{belegt}. Grund: Schlüssel $k'$, die nach $k$ eingefügt wurden, könnten sonst nicht wiedergefunden werden (vgl.~\cite[203]{OW17d}).\\

\subsection{Einfügen von Schlüsseln bei geschlossenem Hashing}
Soll ein Schlüssel $k$ eingefügt werden, wird zunächst die komplette Liste durchsucht, ob der Schlüssel bereits enthalten ist.\\
Wird dabei eine Speicherzelle getroffen, die als \textit{gelöscht} markiert ist, wird diese zum Einfügen vorgemerkt.\\
\begin{itemize}
    \item Ist der Schlüssel $k$ bereits vorhanden, wird er nicht neu eingefügt (keine Duplikate!)
    \item Ist der Schlüssel $k$ nicht vorhanden, wird er in die erste als gelöscht markierte Speicherzelle eingefügt (sofern durch die Sondierungsfunktion getroffen).
    Hierdurch wird die \textit{gelöscht}-Markierung aufgehoben.
    \item ansonsten wird der Schlüssel in das erste freie Feld eingefügt
\end{itemize}

\subsection{Laufzeit}

\begin{itemize}
    \item $m$: Anzahl der Buckets
    \item $n$: Anzahl der zu speichernden Einträge
\end{itemize}

\subsubsection{Offenes Hashing}

\noindent
Die durchschnittliche Listenlänge ist $\frac{n}{m}$.

\begin{itemize}
    \item \textit{insert} / \textit{delete} / \textit{member}: $O(1 + \frac{n}{m}) = O(\frac{n}{m})$ (falls $n \sim m$: $O(\frac{n}{m}) = 1$)
    \item Worst-case: alle Schlüssel in einem Bucket mit Listenlänge $n$ $\rightarrow O(n)$
    \item Platzbedarf: $O(n + m)$
\end{itemize}

\subsubsection{Geschlossenes Hashing}

Abhängig von Lastfaktor\footnote{wenn der Lastfaktor $1$ ist (alle Buckets belegt) müssen im Worst-Case alle $m$ Buckets nach einem Schlüssel durchsucht werden (vgl.~\cite[300]{CL22})} und der Wahl der Hashfunktion.

\begin{itemize}
    \item Best-case: $O(1)$
    \item Worst-case: $O(n)$
\end{itemize}