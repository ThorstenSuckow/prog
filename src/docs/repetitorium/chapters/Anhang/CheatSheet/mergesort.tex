\section{Merge-Sort}


\subsection{Eigenschaften}
\begin{itemize}
    \item stabil
    \item nicht in place
    \item optimal
\end{itemize}

\subsection{Methode}
Der Algorithmus teilt die Eingabefolge rekursiv in zwei gleich-große Folgen auf, bis $n$ ein-elementige Folgen vorhanden sind.\\
Im Anschluss werden die Folgen miteinander verschmolzen, und zwar so, dass die miteinander verschmolzenen Folgen in sortierter Reihenfolge vorliegen.\\


\subsection{Implementierung}
\begin{minted}[
    fontsize=\small
]{java}
    mergeSort(int[] arr, int start, int end) {
        if (arr.length == 1) {
            return arr;
        }

        int mid = (start + end) / 2;
        int[] left = divide(arr, start, mid);
        int[] right = divide(arr, mid + 1, end);

        left = mergeSort(left, 0, left.length - 1);
        right = mergeSort(right, 0, right.length - 1);

        arr = merge(arr, left, right, start);

        return arr;
    }
\end{minted}


\subsection{Laufzeit}
\begin{itemize}
    \item \textbf{Laufzeit}: $O(n\ log\ n)$
\end{itemize}

\subsection{Anmerkungen}
Der Baum hat eine Höhe von $\sim log_2\ n$, auf jeder Ebene werden $O(n)$ Operationen (\textit{merge} oder \textit{divide}) durchgeführt.\\
Die Rekursionstiefe ist bei \textbf{Merge-Sort} logarithmisch beschränkt mit $\lceil log\ n \rceil$, im Gegensatz zu \textbf{Quicksort}.