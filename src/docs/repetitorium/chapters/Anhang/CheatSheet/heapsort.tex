\section{Heapsort}

\subsection{Eigenschaften}
\begin{itemize}
    \item nicht stabil
    \item in place
    \item optimal
\end{itemize}

\subsection{Methode}
Die Eingabefolge wird in einen \textbt{Heap} umgewandelt.\\
Da in einem Heap immer der größte Schlüssel am Anfang der Folge steht, wird der Schlüssel mit dem letzten Element der Folge ausgetauscht und die Heap-Größe um $1$ verkleinert.\\
Das an Position $1$ stehende Element wird nun über ein \textbf{top-down reheapify} in dem Heap an seine eigentliche Position gebracht, damit die Heap-Bedingungen hergestellt sind und das größte Element wieder an Position $1$ steht.\\
Das wird so lange wiederholt, bis der Heap nur noch aus einem Element besteht, dann ist die resultierende Folge sortiert.




\subsection{Laufzeit}
\begin{itemize}
    \item \textbf{Aufbau des Heaps}: $O(n)$
    \item \textbf{sink}: $n$ Aufrufe, jeweils $O(log\ n)$
    \item \textbf{Laufzeit}: $O(n\ log \n)$
\end{itemize}