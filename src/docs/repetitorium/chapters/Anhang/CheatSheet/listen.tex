\section{Listen}

\subsection{Laufzeiten}
\setlength{\tabcolsep}{0.5em}
\renewcommand{\arraystretch}{1.5}%
\begin{table} %[hbtp]
    \centering
    \begin{tabular}{|l | c | c | c|}
        \hline
        & verkettete Liste & doppelt verkettete Liste & Liste als Array \\
        \hline
        \textit{suchen}     &    $O(n) $    &    $O(n)$      & $O(n)$ \\
        \hline
        \textit{einfügen}   &    $O(n) $    &    $O(1)$      & $O(n)$ \\
        \hline
        \textit{entfernen}  &    $O(n) $    &    $O(1)$     &  $O(n)$\\
        \hline
    \end{tabular}
    \caption{Laufzeiten für die Operationen \textit{suchen}, \textit{einfügen} und \textit{entfernen} bei verschiedenen Listenimplementierungen.}
\end{table}

\subsection{Implementierung}

\subsection{Implementierungsbeispiele}

Die folgenden Implementierungsbeispiele orientieren sich an den Praktikumsunterlagen: Dort wurde kein Dummy-Element für
den Kopf der Liste verwendet.

\begin{itemize}
    \item Einfügen eines neuen Elements am Kopf einer einfach verketteten Liste:
\begin{minted}[fontsize=\small]{java}
public void prependHead(Object data) {
    ListElement element = new ListElement(data);
    element.next = head;
    head = element;
}
\end{minted}
    \item[]
    \item Einfügen eines neuen Elements hinter einem angegebenen Element:
\begin{minted}[fontsize=\small]{java}
public void insertAt(ListElement ref, Object data) {
    if (ref == null) {
        throw new IllegalArgumentException();
    }

    ListElement element = new ListElement(data);
    element.next = ref.next;
    ref.next = element;
}
\end{minted}
    \item[]
    \item Einfügen eines Elements vor einem angegebenen Element:
\begin{minted}[fontsize=\small]{java}
public void insertBefore(ListElement ref, Object data) {
    if (ref == null) {
        throw new IllegalArgumentException();
    }

    if (ref == head) {
        prependHead(data);
        return;
    }

    ListElement prev = head;
    while (prev != null && prev.getNext() != ref) {
        prev = prev.getNext();
    }
    if (prev == null) {
        throw new IllegalArgumentException(
            "ref does not seem to be in this list"
        );
    }

    ListElement element = new ListElement(data);
    element.next = ref;
    prev.next = element;
}
\end{minted}
    \item[]
    \item Entfernen eines Elements:
    \begin{minted}[fontsize=\small]{java}
    public void delete(ListElement del) {
        if (ref == null) {
            throw new IllegalArgumentException();
        }

        ListElement prev = getPredecessor(del);
        if (prev == null) {
            head = del.next;
            return;
        }

        prev.next = del.next;
    }
    \end{minted}
    \item[]
    \item Anhängen eines Elements an das Ende (\code{tail}) einer Liste:
\begin{minted}[fontsize=\small]{java}
public void append(Object data) {
    ListElement element = new ListElement(data);

    if (tail == null) {
        head = element;
        head = tail;
        return;
    }

    tail.next = element;
    tail = element;
}
\end{minted}
\end{itemize}