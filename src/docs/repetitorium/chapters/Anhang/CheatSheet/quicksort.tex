\section{Quicksort}


\subsection{Eigenschaften}
\begin{itemize}
    \item nicht stabil
    \item in place
\end{itemize}

\subsection{Methode}
Bei \textbf{Quicksort}  wird ein Schlüssel aus dem Feld als \textbf{Pivot-Element} $p$ gewählt (bspw. das rechte einer Teilfolge).\\
Alle Schlüssel, die kleiner als das Pivotelement sind, werden in die eine Hälfte geschrieben, alle die größer sind, in die andere.
Das Pivotelement selber wird zwischen die beiden Hälften geschrieben, die dann durch Rekursion wiederholt bearbeitet werden, bis die Teilfelder die Größe $1$ haben, womit das gesamte Feld als sortiert gilt.


\subsection{Implementierung}
\begin{minted}[
    fontsize=\small
]{java}
    quicksort (int[] arr, int l, int r) {
        if (l >= r) {
            return;
        }
        int i = l;
        int j = r - 1;
        int pivot = arr[r];
        while (i <= j) {
            while (i < r && arr[i] <= pivot) {
                i++;
            }
            while (j >= 0 && arr[j] > pivot) {
                j--;
            }
            if (i < j) {
                int tmp = arr[i];
                arr[i] = arr[j];
                arr[j] = tmp;
            }
        }
        arr[r] = arr[i];
        arr[i] = pivot;

        quicksort(arr, l, i - 1);
        quicksort(arr, i + 1, r);
    }
\end{minted}


\subsection{Laufzeit}
\begin{itemize}
    \item \textbf{Laufzeit}: $O(n^2)$
\end{itemize}

\subsection{Anmerkungen}
\textbf{Quicksort} nutzt \textbf{divide-and-conquer}, um ein Feld zu sortieren.\\
Die wesentliche Arbeit erfolgt hier in den Partitionierungs-Schritten.\\

\noindent
Im \textbf{worst-case} hat Quicksort eine Laufzeit von $O(n^2)$, allerdings wird in der Literatur darauf hingewiesen, dass das Sortierverfahren eins der schnellsten Sortierverfahren ist, da es im Mittel mit $O(n\ log\ n)$ sortiert.\\
Die \textbf{Effizienz} von Quicksort hängt im Wesentlichen von der Wahl des Pivot-Elementes als auch von der \textit{Rekursionstiefe} ab.