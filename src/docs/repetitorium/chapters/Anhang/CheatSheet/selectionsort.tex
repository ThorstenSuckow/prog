\section{Selection-Sort}

\textit{Sortieren durch Auswahl}

\subsection{Eigenschaften}
\begin{itemize}
    \item nicht stabil
    \item in place
\end{itemize}

\subsection{Methode}
Das Sortierverfahren vergleicht zwei Teilfolgen des zu sortierenden Feldes miteinander.\\
Hierbei ist eine Teilfolge sortiert, die andere unsortiert.
Die sortierte Teilfolge ist zunächst leer.\\
Das jeweils kleinste Element aus der \textit{unsortierten} Teilfolge wird an das Ende der sortierten Teilfolge eingefügt.\\
Das ganze wird so lange wiederholt, bis die unsortierte Teilfolge nur noch aus einem Element besteht, danach ist das Feld sortiert.



\subsection{Implementierung}
\begin{minted}[
    fontsize=\small
]{java}
    for (int i = 0; i < n - 1; i++) {
        int minIndex = i;
        for (int j = i + 1; j < n; j++) {
            if (arr[j] < arr[minIndex]) {
                minIndex = j;
            }
        }
        int tmp = arr[minIndex];
        arr[minIndex] = arr[i];
        arr[i] = tmp;
    }
\end{minted}


\subsection{Laufzeit}
\begin{itemize}
    \item \textbf{Anzahl der Vergleiche}: $\frac{n * ( n - 1)}{2}$
    \item \textbf{Vertauschungen}: $O(n)$
    \item \textbf{worst-case}: $O(n^2)$
    \item \textbf{average-case}: $O(n^2)$
    \item \textbf{best-case}: $O(n^2)$
\end{itemize}