
%% Template for ENG 401 reports
%% by Robin Turner
%% Adapted from the IEEE peer review template

%
% note that the "draftcls" or "draftclsnofoot", not "draft", option
% should be used if it is desired that the figures are to be displayed in
% draft mode.

\documentclass[peerreview]{IEEEtran}
\usepackage{url} % Provides better formatting of URLs.
\usepackage[utf8]{inputenc} % Allows Turkish characters.
\usepackage{booktabs} % Allows the use of \toprule, \midrule and \bottomrule in tables for horizontal lines
\usepackage{graphicx}

\usepackage[backend=biber, style=alphabetic, isbn=true, doi=true]{biblatex}
\usepackage{csquotes}
\hyphenation{op-tical net-works semi-conduc-tor} % Corrects some bad hyphenation

\addbibresource{literatur.bib}


\usepackage{listings}
\usepackage{xcolor}

\definecolor{codegray}{gray}{0.9}
\definecolor{commentgreen}{rgb}{0,0.6,0}
\definecolor{keywordblue}{rgb}{0.2,0.2,0.8}

\lstdefinestyle{javastyle}{
    backgroundcolor=\color{white},
    commentstyle=\color{commentgreen}\itshape,
    keywordstyle=\color{keywordblue}\bfseries,
    numberstyle=\tiny\color{gray},
    stringstyle=\color{red},
    basicstyle=\ttfamily\small,
    breaklines=true,
    captionpos=b,
    keepspaces=true,
    numbers=left,
    numbersep=5pt,
    showspaces=false,
    showstringspaces=false,
    showtabs=false,
    tabsize=2,
    language=Java
}

\usepackage{amsmath,amsthm}
\theoremstyle{definition} % Kein Kursiv, normaler Text
\newtheorem{theorem}{Theorem}
\newtheorem{lemma}[theorem]{Lemma} % zählt mit theorem mit
\newtheorem{definition}[theorem]{Definition}

\begin{document}
%\begin{titlepage}
% paper title
% can use linebreaks \\ within to get better formatting as desired
\title{Runtime Analysis of Shellsort \\with Emphasis on Worst-Case Complexity}


% author names and affiliations

\author{Thorsten Suckow-Homberg \\
Master's Student M.C.Sc.\\
Trier University of Applied Science
}
\date{01.06.2025}

% make the title area
\maketitle
\tableofcontents
\listoffigures
\listoftables
%\end{titlepage}

\IEEEpeerreviewmaketitle



\chapter{Konstruktoren II}\label{konstruktoren}

\section*{Lösung}

\begin{itemize}
    \item Als Standard-Konstruktor wird jeder Konstruktor ohne Argumente bezeichnet.
    \item Ein Konstruktor-Aufruf wird durch den new-Operator ausgelöst.
\end{itemize}


\section*{Anmerkungen und Ergänzungen}

Konstruktoren werden im Skript ab Seite 163 behandelt.\\

Zunächst müssen wir verstehen, was der Kurs mit \textbf{Standardkonstruktor} meint: Dieser wird hier gleichgesetzt
mit einem parameterlosen Konstruktor und stimmt von der Methodensignatur her überein mit dem \textbf{default constructor},
den der Java Compiler zur Verfügung stellt, wenn kein expliziter Konstruktor implementiert ist.\\

\textit{Bloch} schreibt zu dem \textbf{default constructor} \footnote{
    Java Language Specification - 8.8.9. Default Constructor: \url{https://docs.oracle.com/javase/specs/jls/se21/html/jls-8.html#jls-8.8.9}
}:

\blockquote[{\cite[19 f.]{Blo17}}]{
    In the absence of explicit constructors, however, the compiler provides a public, parameterless default constructor [...]
    A default constructor is generated only if a class contains no explicit constructors
}

Bei \textit{Ullenboom} findet sich:

\blockquote[{\cite[515]{Ull12}}]{
    Wenn wir in unserer Klasse überhaupt keinen Konstruktor angeben, legt der Compiler automatisch einen an. Diesen
    Konstruktor nennt die Java Sprachdefinition (JLS) default constructor, was wir als vorgegebener Konstruktor
    (selten auch Vorgabekonstruktor) eindeutschen wollen."
}

Der Autor merkt im folgenden bzgl. der Begrifflichkeit an:

\blockquote[{\cite[516]{Ull12}}]{
    In der Java Language Specification gibt es bei den Konstruktoren nur die Trennungen in no-arg-constructor
    (parameterloser Konstruktor) und default constructor (vorgegebener Konstruktor), aber den Begriff
    »standard constructor« gibt es nicht.
}

sowie weiter:

\blockquote[{\cite[517]{Ull12}}]{
    Einige Autoren nennen nur den vom Entwickler explizit geschriebenen parameterlosen Konstruktor »Standard-Konstruktor«
    und trennen dies sprachlich von dem Konstruktor, den der Compiler generiert hat, den sie weiterhin »Default-Konstruktor« nennen.
}

Anmerkungen zur Verwendung der Begrifflichkeiten im Kurs finden sich auch in der Aufzeichnung des ersten Online-Tutoriums
vom 07.10.2023, ab 03:39 - 03:41 (hh:mm).\\

Wir dürfen also davon ausgehen, dass der Standardkontruktor im Kurs gleichbedeutend zu einem \underline{parameterlosen}
Konstruktor ist.\\

Zunächst ist festzuhalten, dass Konstruktoren in Java formal nicht mit Objekt- /Klassenmethoden gleichzusetzen sind
und nicht \unerline{nicht zu den vererbbaren Eigenschaften} einer Klasse zählen, wie bspw. (je nach Sichtbarkeit) Methoden oder Attribute\footnote{
    The Java Tutorials - Inheritance: \url{https://docs.oracle.com/javase/tutorial/java/IandI/subclasses.html}
} (ein Zugriff über \code{super} auf den Konstruktor einer Elternklasse ist trotzdem möglich).\\

Wenn für eine Klasse kein Konstruktor definiert wurde, stellt der Compiler einen \textbf{default constructor}
zur Verfügung\footnote{
    The Java Language Specification - 8.8.9. Default Constructor: \url{https://docs.oracle.com/javase/specs/jls/se21/html/jls-8.html#jls-8.8.9}↩
}.
Die Parameterliste des default constructors ist leer.
Sie entspricht damit der Signatur des Standardkonstruktors.

\begin{lstlisting}[language=java]
class A {

    // implizit deklarierter default constructor
    // public A() {
    // }

    public static void main(String[] args) {
        A a = new A(); // ruft den implizit deklarierten default constructor
                       // auf
    }

}
\end{lstlisting}

Wenn ein \underline{explizit} deklarierter Konstruktor vorhanden ist, stellt der Compiler keinen default constructor zur Verfügung.\\

Im folgenden Beispiel führt \code{new A()} zu einem Compiler-Fehler, da kein parameterloser Konstruktor explizit oder implizit für \code{A} deklariert ist.

\begin{lstlisting}[language=java]
class A {

    int a;

    // explizit deklarierter Konstruktor
    // unterbindet den impliziert deklarierten default constructor
    public A(int value) {
        a = value;
    }

    public static void main(String[] args) {
        A a1 = new A(i); // ruft den explizit deklarierten Konstruktor auf
        A a2 = new A();  // fuehrt zu einem Compiler-Fehler
    }

}
\end{lstlisting}

Wenn ein Konstruktor nicht \underline{explizit} einen Konstruktor seiner Elternklasse aufruft, ruft er \underline{implizit} \code{super()} auf.

Wenn die Elternklasse in solchen Fällen keinen parameterlosen Konstruktor deklariert hat, wird ein Compiler-Fehler erzeugt.

Im folgenden Beispiel erweitert \code{B} die Klasse \code{A}. \code{B} hat einen Konstruktor deklariert, dessen
formale Parameterliste \code{int} ist. Dieser Konstruktor ruft implizit \code{super()} auf:

\begin{lstlisting}[language=java]
class A {

}

class B {

    int val;

    public B (int a) {
        // implizit aufgerufen:
        // super();
        val = a;
    }

    public static void main(String[] args) {
        B b = new B(1);
    }

}
\end{lstlisting}

Ergänzend sei daran erinnert, dass jede Klasse in Java direkt oder indirekt von \code{java.lang.Object} erbt\footnote{
    The Java Tutorials - Object as a Superclass: \url{https://docs.oracle.com/javase/tutorial/java/IandI/objectclass.html}↩
}: Der Konstruktor der Klasse \code{java.lang.Object} ist also in einer Konstruktor-Aufrufkette enthalten.\\

Im nächsten Beispiel erbt \code{C}  von \code{B}. Ein default constructor wird implizit für \code{C} deklariert. Allerdings
führt der Versuch, ein Objekt vom Typ \code{C} zu erzeugen, zu einem Compiler-Fehler, da \code{B} keinen parameterlosen
Konstruktor - den Standardkonstruktor - deklariert:


\begin{lstlisting}[language=java]
class C extends B {


    public static void main(String[] args) {
        C c = new C(); // Compiler-Fehler
    }

}
\end{lstlisting}

Damit ein Objekt vom Typ \code{C} erzeugt werden kann, benötigt \code{C} einen Standardkonstruktor mit einem expliziten
Aufruf des Konstruktors von \code{B} - da \code{B} selber keinen Standardkonstruktor deklariert hat, muss der Konstruktor
mit der Signature \code{B(int)} aufgerufen werden:

\begin{lstlisting}[language=java]
class C extends B {

    public C() {
       super(2); // ruft den Konstruktor von B mit dem Argument 2 auf
    }

    public static void main(String[] args) {
        C c = new C();
    }

}
\end{lstlisting}

Wir sehen, dass in Java bei einem explizit deklarierten Konstruktor als \underline{erstes}
der Konstruktor der Elternklasse aufgerufen wird (implizit durch den Compiler) bzw. aufgerufen werden muss (falls explizit implementiert)\footnote{
    The Java Language Specification - 8.8.7. Constructor Body: \url{https://docs.oracle.com/javase/specs/jls/se21/html/jls-8.html#jls-8.8.7}
}.\\

Es folgen ergänzende Beispiele. Details sind den Quelltextkommentaren zu entnehmen.

\begin{lstlisting}[language=java]
class A {
    // impliziter default constructor: vorhanden

    // impliziter default constructor: ruft Konstruktor von java.lang.Object
    // auf
}


class B extends A {

    // impliziter default constructor: vorhanden

    // impliziter default constructor: ruft Konstruktor von A auf
}


class C {

    // impliziter default constructor: nicht vorhanden

    public C() {
        // impliziter Aufruf von java.lang.Object's Konstruktor
        System.out.println("C created.");
    }
}


class D extends C {

    // impliziter default constructor: vorhanden

    // impliziter default constructor: ruft Konstruktor von C auf

}


class E extends D {

    // impliziter default constructor: nicht vorhanden

    public E() {
        // impliziter Aufruf von D's Konstruktor
        System.out.println("E created.");
    }
}


class F extends E {

    // impliziter default constructor: nicht vorhanden

    public F(int x) {
        // impliziter Aufruf von E's Standardkonstruktor
        System.out.println("F created");
    }
}


class G extends F {

    // impliziter default constructor: nicht vorhanden

    public G(int x) {
        super(x); // expliziter Aufruf von F's Konstruktor

        // Das Auskommentieren der o.a. Anweisung fuehrt zu einem impliziten
        // Aufruf von 'super()', was einen Compiler-Fehler produziert:
        // Da kein Standardkonstruktor in F deklariert ist, muessen wir dem
        // Konstruktor explizit mitteilen, welcher Konstruktor der
        // Elternklasse aufgerufen werden soll.
        System.out.println("G created");
    }
}
\end{lstlisting}





%\bibliographystyle{geralpha}			% Literaturverzeichnis
%\bibliography{literatur}     			% BibTeX-File literatur.bib
%\raggedright
\sloppy
\printbibliography
\fussy








\end{document}
