\section{Introduction}

\textbf{Shellsort}, introduced by \textit{Donald Shell} in 1959 (\cite[]{She59}), is a generalization of insertion sort\footnote{
see \cite[250]{SW11} for a brief introduction on insertion sort
} that improves performance by initially sorting elements far apart, yielding optimized arranged sequences of keys for subsequent passes.
The algorithm uses a sequence of decrementing gaps, comparing elements separated by a distance of $h_t$.\footnote{
hence, Shellsort is also referred to as ``diminishing increment sort`` (\cite[88]{OW17b})
}.
The general idea is to use \blockquote[{\cite[48]{CL22}}]{
    insertion sort on periodic subarrays of the input
    to produce a faster sorting algorithm.
}

\subsection{Algorithm Description and Notation}
We use the Shellsort implementation shown in Listing~\ref{lst:shell}.
In the following, we provide a brief description of the algorithm, mostly based on~\cite[84]{Knu97b}.
Throughout this document, the binary logarithm $\log_2$ is abbreviated as $\lg$.


Given an input of size $n$, the implementation based on a halving sequence of the form $h_t = \lfloor \frac{n}{2^t} \rfloor$ performs approximately $\lg\ n$ passes.
For this, an \textbf{increment sequence} $h$ is defined that determines the distance between two elements compared during sorting.
For instance, for $n = 16$, we choose $t=\lg n = 4$, resulting in the sequence $(8, 4, 2, 1)$ with

\[
    h_{3} = \lfloor\frac{16}{2^1}\rfloor = 8, h_{2} = \lfloor\frac{16}{2^2}\rfloor = 4, \ldots , h_{0} = \lfloor\frac{16}{2^4}\rfloor = 1
\]

such that

\[
    h_i = \lfloor {\frac{n}{2^{t-i}}} \rfloor
\]

for $0 \leq i \leq t - 1$.\\

In the first pass, the algorithm compares and sorts elements that are $h_{3} = \frac{16}{2^{4-3}} = 8$ positions apart, while the final pass processes keys $h_0 = \frac{16}{2^{4-0}} = 1$ positions apart, effectively performing \textit{insertion sort}.\\

To further illustrate runtime behavior, we define counting variables $c_1, c_2, c_3, c_4$.

\vspace{4mm}
\begin{lstlisting}[style=javastyle, caption={Shellsort implementation using halving sequences. }, label=lst:shell]
public static int[] sort(int[] arr) {
    int n = arr.length;
    int delta = n/2;
    int min;
    int j;
    int len = arr.length;

    while (delta > 0) { // c1
        for (int i = delta; i < len; i++) {
            // c2
            min = arr[i];
            j = i;
            while (j - delta >= 0 &&
                   min < arr[j - delta]) {
                // c3;
                arr[j] = arr[j - delta];
                j -= delta;
            }
            arr[j] = min;
        }
        delta = delta / 2;
    }
    return arr;
}
\end{lstlisting}
\vspace{4mm}


\subsection{Role of the Increment Sequence Regarding Complexity}

As demonstrated by \textit{Knuth} in~\cite[91]{Knu97b}, the performance of Shellsort heavily  depends on the chosen increment sequence $h$: In particular, an upper bound of $O(n^{\frac{3}{2}})$ can be achieved\footnote{
\textit{Knuth} refers to the work of \textit{Papernov and Stasevich} \cite{PS65} in this context.
}, if the sequence satisfies

\[
    h_s = 2^{s+1} - 1, 0 \leq s < t = \lfloor{\lg n} \rfloor
\]

\noindent
In our case, where a trivial halving sequence is used, we conservatively assume an upper bound of $O(n^2)$.
