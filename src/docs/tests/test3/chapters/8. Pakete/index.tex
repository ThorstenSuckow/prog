\chapter{Pakete}

\section*{Lösung}

\begin{itemize}
    \item Auf Elemente, die in einer Klasse als public deklariert sind, kann aus allen Klassen aller Pakete zugegriffen werden.
    \item Auf Elemente, für die kein Modifizierer angegeben ist, kann aus allen Klassen desselben Pakets zugegriffen werden (\textit{package private}).
\end{itemize}


\section*{Anmerkungen und Ergänzungen}

Zugriffsmodifizierer werden ausführlich im Skript auf Seite 39 behandelt.
\\

Es lohnt, sich die Zugriffstabelle\footnote{
    im Skript auf Seite 225, ausserdem bei ``The Java™ Tutorials - Controlling Access to Members of a Class``: \url{https://docs.oracle.com/javase/tutorial/java/javaOO/accesscontrol.html}
} einzuprägen:

\setlength{\tabcolsep}{1.5em}
\renewcommand{\arraystretch}{1.5}%
\begin{table} %[hbtp]
    \centering
    \begin{tabular}{c | c | c | c | c}
        \textbf{Modifizierer} & Klasse & Package & Unterklasse & Welt \\
        \hline
        \code{public}   & $\checkmark$   & $\checkmark$ & $\checkmark$ & $\checkmark$   \\
        \code{protected}   & $\checkmark$   & $\checkmark$ & $\checkmark$ & $-$   \\
        kein Modifizierer   & $\checkmark$   & $\checkmark$ & $-$ & $-$   \\
        \code{private}   & $\checkmark$   & $-$ & $-$ & $-$   \\
    \end{tabular}
    \caption{Zugriffsmodifizierer und ihre Sichtbarkeiten}
    \label{tab:modifier}
\end{table}

Wenn kein expliziter Zugriffsmodifizierer angegeben wurde, ist die Sichtbarkeit implizit \textit{package-private} (Zeile 3 in Tabelle~\ref{tab:modifier}).

Erwähnenswert ist ausserdem, dass Java so etwas wie ``innere Pakete`` nicht kennt.
Die Java Tutorials erklären dazu, dass das Package-Konzept kein hierarchisches Modell repräsentiert:\blockquote{
    At first, packages appear to be hierarchical, but they are not.\footnote{
        ``The Java™ Tutorials - Apparent Hierarchies of Packages``: \url{https://docs.oracle.com/javase/tutorial/java/package/usepkgs.html}
    }
}
Sie verdeutlichen dies Anhand des Beispiels von \code{import}-Deklarationen, die Wildcards (``*``) in Verbindung mit einem qualifizierten Namen
nutzen\footnote{
    Eine sog. ``Type-Import-on-Demand Declaration``. Siehe  ``Java Language Specification - 7.5.2. Type-Import-on-Demand Declarations``: \url{https://docs.oracle.com/javase/specs/jls/se21/html/jls-7.html#jls-7.5.2}
}

So können alle öffentlichen Klassen aus dem Paket \code{java.awt} in Typdeklarationen einer Übersetzungseinheit\footnote{
    ``Compilation Unit``. Siehe hierzu ``Java Language Specification - 7.3. Compilation Units``: \url{https://docs.oracle.com/javase/specs/jls/se8/html/jls-7.html#jls-7.3}
} über ihren Namen referenziert werden, wenn wie folgt importiert wird:

\begin{lstlisting}[language=java]
import java.awt.*;
\end{lstlisting}

Dies schließt allerdings \unerline{nicht} die Klassen ein, die in Paketen liegen, die ihrem qualifizierten Namen nach scheinbar ``in`` \code{java.awt} liegen,
wie bspw. \code{java.awt.color}. Diese müssen separat importiert werden:

\begin{lstlisting}[language=java]
import java.awt.*;
import java.awt.color.*;
\end{lstlisting}
