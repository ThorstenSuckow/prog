\chapter{Modifikatoren}

\section*{Lösung}

\begin{itemize}
    \item Eine Methode kann nicht private abstract erklärt werden.
    \item Von Klassen, die als abstract gekennzeichnet sind, können keine Exemplare erzeugt werden.
    werden.
\end{itemize}


\section*{Anmerkungen und Ergänzungen}

Im Folgenden werden die Antworten anhand der Sprachspezifikationen hergeleitet bzw. widerlegt.

\begin{itemize}
    \item Eine Methode kann nicht private abstract erklärt werden.
\end{itemize}

Eine Implementierung einer als \code{private} markierten abstrakten Methode in einer Kindklasse wäre u.a. aus folgenden Gründen \underline{nicht} möglich.

Die Sprachspezifikationen legen fest:\blockquote{
It is a compile-time error to declare an abstract class type such that it is not possible to create a subclass that implements all of its abstract methods.\footnote{
    ``Java Language Specification - 8.1.1.1. abstract Classes``:\url{https://docs.oracle.com/javase/specs/jls/se21/html/jls-8.html#jls-8.1.1.1}
}
}


Eine abstrakte Klasse stellt  eine unvollständige Klasse dar\footnote{
``An abstract class is a class that is incomplete, or to be considered incomplete.`` in ``Java Language Specification - 8.1.1.1. abstract Classes``: \url{8.1.1.1. abstract Classes https://docs.oracle.com/javase/specs/jls/se21/html/jls-8.html#jls-8.1.1.1
}}.
Unterklassen müssen entweder selber als \code{abstract} gekennzeichnet werden oder die abstrakten Methoden implementieren.

Folglich führt eine als \code{private} markierte abstrakte Methode zu dem Compilerfehler\footnote{Fehlermeldung generiert durch JDK21.0.0}

\begin{lstlisting}[language=text]
illegal combination of modifiers: abstract and private
\end{lstlisting}


\begin{itemize}
    \item Von Klassen, die als abstract gekennzeichnet sind, können keine Exemplare erzeugt werden.
\end{itemize}

Die Sprachspezifikation legt fest:\blockquote{
    It is a compile-time error if an attempt is made to create an instance of an abstract class using a class instance creation expression.\footnote{
        ``Java Language Specification - 8.1.1.1. abstract Classes``: \url{https://docs.oracle.com/javase/specs/jls/se21/html/jls-8.html#jls-8.1.1.1}
    }
}

\begin{itemize}
    \item Von Klassen, die als abstract gekennzeichnet sind, können keine Unterklassen gebildet werden.
\end{itemize}

Abstrakte Klassen sind unvollständige Klassen.
Es \underline{muss} deshalb möglich sein, Unterklassen von ihnen zu erzeugen.
Die Sprachspezifikationen legen hierzu fest: \blockquote{
    The declaration of an abstract method m must appear directly within an abstract class (call it A).\footnote{
      ``Java Language Specification - 8.4.3.1. abstract Methods``: \url{https://docs.oracle.com/javase/specs/jls/se21/html/jls-8.html#jls-8.4.3.1}
    }
}

und weiter \blockquote{
    Every subclass of A that is not abstract[...] must provide an implementation for m, or a compile-time error occurs.\footnote{
       ebenda
    }
}


\begin{itemize}
    \item  Eine final erklärte Klasse darf keine abstract erklärten Methoden haben.
\end{itemize}

Von als \code{final} deklarierte Klassen können keine Unterklassen gebildet werden\footnote{
siehe hierzu Skript Seite 38, Absatz ``final``.
}.

Wenn die Klasse abstrakte Methoden beinhaltet, muss die Klasse als \code{abstract} implementiert werden.
Dies schließt die gleichzeitige Verwendung von \code{final}
aus:\blockquote{
    It is a compile-time error if a class is declared both final and abstract, because the implementation of such a class could never be completed (§8.1.1.1).\footnote{
        ``Java Language Specification - 8.1.1.2. sealed, non-sealed, and final Classes``: \url{https://docs.oracle.com/javase/specs/jls/se21/html/jls-8.html#jls-8.1.1.2}
    }
}
