\chapter{Vererbung}

\section*{Lösung}

\begin{itemize}
    \item Eine Variable vom Typ B kann eine Instanz von A referenzieren.
    \item Die Klasse A erbt alle nicht private deklarierten Objektattribute von B.
    \item Ein Exemplar von A kann auf alle nicht private deklarierten Objektattribute von B zugreifen.
    \item Geerbte Klassenattribute sind in Klasse A und B identisch.
\end{itemize}


\section*{Anmerkungen und Ergänzungen}

Die Lösungen zu den Fragen lassen sich aus dem Skriptinhalt leicht herleiten:

\begin{itemize}
    \item ``Eine Variable vom Typ B kann eine Instanz von A referenzieren``.
\end{itemize}

Der Sachverhalt wird im Skript zu Begin des Abschnitts ``8.4 Zuweisungskompatibilität`` erklärt.

\begin{itemize}
    \item ``Die Klasse A erbt alle nicht private deklarierten Objektattribute von B``.
\end{itemize}
Siehe hierzu das Beispiel in der Sprachspezifikation\footnote{``Java Language Specification - Example 8.2-4. Inheritance of private Class Members``: \url{https://docs.oracle.com/javase/specs/jls/se21/html/jls-8.html#d5e13702}},
im offiziellen Java Tutorial\footnote{``The Java™ Tutorials - Inheritance``: \url{https://docs.oracle.com/javase/tutorial/java/IandI/subclasses.html}}
und im Skript auf Seite 224 ``Der Modifizierer private``.

\begin{itemize}
    \item ``Ein Exemplar von A kann auf alle nicht private deklarierten Objektattribute von B zugreifen.``
\end{itemize}

Ein ``Exemplar`` von \code{A} ist ein Objekt vom Typ \code{A}.
Das Exemplar kann auf geerbte Objektattribute zugreifen:

\begin{lstlisting}[language=java]
package programm;

class B {
    int foo = 2;
}

class A extends B {

    public int getFoo() {
        return foo;
    }
}
\end{lstlisting}


Ist das Objektattribut \code{protected} oder \code{package private}, kann auch \underline{über} das Exemplar\footnote{
hier greift Code, der nicht zu der Klasse \textit{A} gehört, auf das Objektattribut zu.
} auf das Objektattribut
innerhalb desselben Package zugegriffen werden:

\begin{lstlisting}[language=java]
package programm;

public class Beispiel {

    public static void main(String[] args) {
        A a = new A();
        System.out.println(a.foo);
    }

}
\end{lstlisting}

Ist das Objektattribut \code{public}, kann von überall her auf das geerbte Attribut zugegriffen werden.


\begin{itemize}
    \item Geerbte Klassenattribute sind in Klasse A und B identisch.
\end{itemize}

Ein Klassenattribut, das für Klasse \code{B} definiert wurde, wird (sofern durch den Zugriffsmodifizierer ermöglicht) auch von Klasse \code{A} geerbt.
Wird der Wert des Klassenattributs geändert, ist diese Änderung in allen Unterklassen sichtbar - egal, in welchem Codeteil die Änderung
implementiert ist.


\begin{lstlisting}[language=java]
class B {
    public static int foo = 2;
}

class A extends B {

    public void changeFoo() {
        foo = 4;
    }

}


public class KlassenAttribut {

    public static void main(String[] args) {
        System.out.println(A.foo); // 2
        B.foo = 3;
        System.out.println(A.foo); // 3
        A a = new A();
        a.changeFoo();
        System.out.println(B.foo); // 4
        System.out.println(A.foo); // 4


    }

}
\end{lstlisting}

Wird ein Klassenattribut in \code{A} angelegt, welches ein Klassenattribut in \code{B} \underline{verdeckt}\footnote{
    im Skript auf Seite 212 ff.
    Im Englischen ist der Begriff hierzu ``hiding``.
    Ein ausführliches Beispiel findet sich in der Sprachreferenz unter ``Java Language Specification - Example 8.4.9-2. Overloading, Overriding, and Hiding``:
    \url{https://docs.oracle.com/javase/specs/jls/se21/html/jls-8.html#d5e15760},
    weitere Hinweise in ``The Java™ Tutorials - Hiding Fields``\url{https://docs.oracle.com/javase/tutorial/java/IandI/hidevariables.html} sowie
    bei Ullenboom in~\cite[483 ff.]{Ull12}.
}, steuert der Aufrufkontext Zuweisung und Rückgabe des Klassenattributs:

\begin{lstlisting}[language=java]
class B {
    public static int foo = 2;
}

class A extends B {
public static int foo = 2;

    public void printFoo() {
        System.out.println(foo);
    }

    public void printParentFoo() {
        System.out.println(super.foo);
        System.out.println(B.foo);
    }

}

public class KlassenAttribut {

    public static void main(String[] args) {
        System.out.println(A.foo); // 2
        B.foo = 3;
        System.out.println(A.foo); // 2
        System.out.println(B.foo); // 3
        A.foo = 1;
        System.out.println(A.foo); // 1
        A a = new A();
        a.printFoo();              // 1
        a.printParentFoo();        // 3\n3
    }

}
\end{lstlisting}