\chapter{Schnittstellen}

\section*{Lösung}

\begin{itemize}
    \item Jede nicht abstrakte Klasse, die eine Schnittstelle implementiert, muss alle Methoden dieser Schnittstelle bereitstellen.
    \item Alle Methoden einer Schnittstelle sind öffentlich.
\end{itemize}

(für beides siehe auch den Exkurs unter ``Anmerkungen und Ergänzungen``)

\section*{Anmerkungen und Ergänzungen}

Das Konzept von Schnittstellen mit Anwendungsbeispielen findet sich im Skript auf Seite 230 ff.

Die in der Aufgabe getroffenen Aussagen lassen sich darüber hinaus über die Sprachspezifikationen herleiten bzw. widerlegen.

\begin{itemize}
    \item Jede Klasse kann nur eine Schnittstelle implementieren.
\end{itemize}

Widerlegt durch das Skript auf Seite 232, Absatz 3, außerdem wird in den Java Tutorials darauf hingewiesen, dass
\blockquote{
    Your class can implement more than one interface, so the implements keyword is followed by a comma-separated list of the interfaces implemented by the class.\footnote{``The Java™ Tutorials - Implementing an Interface``: \url{https://docs.oracle.com/javase/tutorial/java/IandI/usinginterface.html}}
}


\begin{itemize}
    \item Schnittstellen müssen als abstract gekennzeichnet werden.
\end{itemize}

Ein \code{interface} ist in Java implizit \code{abstract}\footnote{im Skript erklärt auf Seite 231, Absatz 2}.
Der Modifizierer darf zwar angegeben werden, aber da er obsolet ist, sollte er nicht verwendet werden\footnote{
``Java Language Specification - 9.1.1. Interface Modifiers``: \url{https://docs.oracle.com/javase/specs/jls/se21/html/jls-9.html#jls-9.1.1}
}

\begin{itemize}
    \item Jede nicht abstrakte Klasse, die eine Schnittstelle implementiert, muss alle Methoden dieser Schnittstelle bereitstellen.
\end{itemize}

Ist die Methode einer Schnittstelle \code{abstract}\footnote{siehe nachfolgender Exkurs}, gilt hier das gleiche Prinzip wie bei den abstrakten Klassen:
Ist eine implementierende Klasse selber nicht abstrakt, muss die implementierende Klasse die Schnittstellenfunktionen zur Verfügung stellen\footnote{
siehe hierzu auch ``Java Language Specification - 9.4.1. Inheritance and Overriding``: \url{https://docs.oracle.com/javase/specs/jls/se21/html/jls-9.html#jls-9.4.1}
}.

\subsection*{Exkurs - Alle Methoden einer Schnittstelle sind öffentlich}

Im Kontext des im Skript enthaltenen Lehrmaterials wird diese Antwort als richtige Antwort gezählt.
Denn ein Interface beschreibt eine Schnittstelle, deren öffentliche Methoden in implementierenden Klassen zur Verfügung gestellt werden müssen.
\\

Allerdings bietet Java seit Version 8 bereits die Möglichkeit, Methoden in Schnittstellen auszuimplementieren, sofern diese mit dem Modfizierer \code{default} deklariert werden\footnote{
    ``The Java™ Tutorials - Default Methods``: \url{https://docs.oracle.com/javase/tutorial/java/IandI/defaultmethods.html}
}.

Seit Java 9 is es außerdem erlaubt, als \code{private} deklarierte Methoden in einem \code{interface} zu implementieren\footnote{
siehe hierzu ``JEP 213: Milling Project Coin`` unter \url{https://bugs.openjdk.org/browse/JDK-8042880} sowie ``Java Language Specification - 9.4. Method Declarations``: \url{https://docs.oracle.com/javase/specs/jls/se21/html/jls-9.html#jls-9.4}
}.
Als \code{private} deklarierte Schnittstellenmethoden erlauben eine weitere Kapselung von Funktionalität, die von den \textit{default}-Methoden
benötigt wird: Werden mehrere default-Methoden implementiert, die Funktionalität wiederverwenden, kann diese Funktionalität in private Methoden ausgelagert werden.

Als einfaches Beispiel sei folgender Code gegeben, der von JDK21.0.0.0 problemlos übersetzt wird:

\begin{lstlisting}[language=java]

interface Fooable {
    default String getFoo() {
        return foo();
    }

    private String foo() {
        return "foo";
    }

}

public class B implements Fooable {
    public static void main(String[] args) {
        B b = new B();
        System.out.println(b.getFoo());
    }
}

\end{lstlisting}

In diesem Zusammenhang  ist die Aussage wie folgt zu verstehen:\\

``Alle \textbf{abstrakten} Methoden einer Schnittstelle sind öffentlich``.\\

Die Sprachspezifikationen halten hierzu fest:
\blockquote{
    If no access modifier is given, the method is implicitly public.
}
sowie
\blockquote{
    An interface method lacking a private, default, or static modifier is implicitly abstract.\footnote{
    beides zu finden in  ``Java Language Specification - 9.4. Method Declarations``: \url{https://docs.oracle.com/javase/specs/jls/se21/html/jls-9.html#jls-9.4}
    }
}