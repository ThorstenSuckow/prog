\chapter{Standardkonstruktor}

Die richtigen Antworten lauten:

\begin{itemize}
    \item Klasse B
    \item Klasse C
\end{itemize}


\section*{Anmerkungen und Ergänzungen}

Die Frage lässt sich mit dem Wissen beantworten, das man bereits bei der Lösung von ``Konstruktoren II`` (Frage ~\ref{konstruktoren}) angewendet hat.\\

Da wir den Standardkonstruktor als \textbf{parameterlosen Konstruktor} verstehen, fällt Klasse \code{A} als
Antwortmöglichkeit weg: Diese Klasse hat einen expliziten Konstruktor mit der Parameterliste \code{int} deklariert,
somit wird kein impliziter \textbf{default constructor} zur Verfügung gestellt.

Dass in Klasse \code{C} ein parameterloser Konstruktor auch Implementierung im Konstruktor-Rumpf vorhält, ändert nichts daran,
dass auch dieser Konstruktor als Standardkonstruktor zu verstehen ist.