\chapter{Vergleichsoperator}

Die richtigen Antworten lauten:

\begin{itemize}
    \item a == null liefert true, wenn a eine Referenzvariable ist, der ein Wert zugewiesen wurde, die aber auf kein Objekt verweist.
    \item Mit dem Operator == wird getestet, ob zwei Referenzvariablen auf dasselbe Objekt verweisen.
    \item Bei primitiven Datentypen liefert a == b den Wert true, wenn beide Variablen denselben Wert haben.
\end{itemize}


\section*{Anmerkungen und Ergänzungen}

Vergleichsoperationen werden im Script ab Seite 77 behandelt.\\

Für die Beantwortung der Frage ist das Wissen um die verschiedenen Typen in Java wichtig, die sich unterteilen in
\textbf{primitive Typen} und \textbf{Referenztypen}\footnote{
    Java Language Specification - Chapter 4. Types, Values, and Variables: \url{https://docs.oracle.com/javase/specs/jls/se21/html/jls-4.html}
}: zu den primitiven Typen~\footnote{
    Java Language Specification - 4.2. Primitive Types and Values: \url{https://docs.oracle.com/javase/specs/jls/se21/html/jls-4.html#jls-4.2}
} gehören bspw. \code{boolean} und \code{double}.

Zu den Referenztypen gehören \textit{Klassen}, \textit{Interfaces}, \textit{Typvariablen}~\footnote{
    Java Language Specification - 4.4. Type Variables: \url{https://docs.oracle.com/javase/specs/jls/se21/html/jls-4.html#jls-4.4}
}
und \textit{Arrays}~\footnote{
    Java Language Specification - 10.1. Array Types \url{https://docs.oracle.com/javase/specs/jls/se21/html/jls-10.html#jls-10.1}
}.\\

Die Java Language Specification legt für \code{null} fest~\footnote{Java Language Specification - 4.1. The Kinds of Types and Values\url{https://docs.oracle.com/javase/specs/jls/se21/html/jls-4.html}}:

\blockquote{
    The null reference can always be assigned or cast to any reference type.
}

Daraus lässt sich schliessen, dass, wenn einer Referenzvariable ein Wert zugewiesen wurde, der auf \underline{kein} Objekt verweist,
dieser Wert nur \code{null} sein kann (im anderen Fall hätte der Compiler bereits einen Fehler angezeigt), weshalb \code{a == null} auch
\code{true} liefert.

Beispielsweise produziert der folgende Code
\begin{lstlisting}[language=java]
class Foo {

    public static void main(String[] args) {
        Foo f = 42;
    }
}
\end{lstlisting}

den Compiler-Fehler

\begin{lstlisting}[language=text]
incompatible types: int cannot be converted to Foo
Foo f = 3;
^
1 error
\end{lstlisting}

Das Verhalten unterscheidet sich je nach verwendeter Programmiersprache: So kennen bspw. Python~\footnote{
    The Python Standard Library - Built-in Types: \url{https://docs.python.org/3/library/stdtypes.html?highlight=null#the-null-object}
} als auch JavaScript~\footnote{JavaScript Reference - Expressions and operators (null): \url{https://developer.mozilla.org/en-US/docs/Web/JavaScript/Reference/Operators/null}
} \code{null} als Objekt.\\

Aufpassen muss man, sobald \textbf{boxing} ins Spiel kommt~\footnote{
    im Skript auf Seite 237; ausserdem unter
    The Java™ Tutorials - Autoboxing and Unboxing: \url{https://docs.oracle.com/javase/tutorial/java/data/autoboxing.html}
}: So ist es durchaus möglich,
einem Objekt einen primitiven Datentypen zuzuweisen. Allerdings sorgt der Compiler dafür, dass der primitive Datentyp
in seine Objektrepräsentation ``verpackt`` wird:

Der folgende Code verdeutlicht dies:

\begin{lstlisting}[language=java]
class Foo {

    public static void main(String[] args) {
        Object x = 42;
        Boolean y = true;
        Character z = 'a';

        System.out.println(x.getClass()); // class java.lang.Integer
        System.out.println(y.getClass()); // class java.lang.Boolean
        System.out.println(z.getClass()); // class java.lang.Character
    }

}
\end{lstlisting}

Umgekehrt ist auch das \textbf{unboxing} möglich. Weitere Informationen und zur Anwendung von \textit{boxing} und \textit{unboxing}
finden sich u.a. in~\cite[732 ff]{Ull12}.\\

Ob Objekte den gleichen Inhalt haben, wird nicht mit dem Vergleichsoperator ``==`` überprüft: Zwar lässt sich bei Referenztypen
so feststellen,ob zwei Variablen das gleiche Objekt referenzieren. Um aber auf den gleichen Inhalt zu überprüfen, kann
bspw. die Methode \textit{equals(Object)}~\footnote{
    Java API Docs - Class Object \url{https://docs.oracle.com/en/java/javase/21/docs/api/java.base/java/lang/Object.html#equals(java.lang.Object)}
} aus \textit{java.lang.Object} überladen (oder überschrieben) werden:\\


\begin{lstlisting}[language=java]
class Foo {

    int x = 0;

    public int getX() {
        return x;
    }

    public void setX(int x) {
        this.x = x;
    }

    // equals(Object) ueberschreiben
    public boolean equals(Object f2) {
        // durch die folgende Überprüfung wird sichergestellt, dass
        // das explizite casten keine Exception wirft
        if (!(f2 instanceof Foo)) {
            return false;
        }
        return this.equals((Foo)f2);
    }

    // equals(Object) mit equals(Foo) ueberladen
    public boolean equals(Foo f2) {
        return this == f2 || this.x == f2.x;
    }
}
\end{lstlisting}\\

Die Antwort ``a == null liefert false, wenn a eine Referenzvariable ist und a noch nicht initialisiert wurde.`` ist falsch,
da eine Referenzvariable ohne explizite Wertzuweisung automatisch den Wert \code{null} hat~\footnote{Im Skript auf Seite 139 vermerkt. Ausserdem in The Java™ Tutorials - Primitive Data Types (Default Values): \url{https://docs.oracle.com/javase/tutorial/java/nutsandbolts/datatypes.html}}