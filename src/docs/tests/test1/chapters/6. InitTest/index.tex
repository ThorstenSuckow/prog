\chapter{InitTest}

Die richtigen Antworten lauten:

\begin{itemize}
    \item t.ok hat den Wert false.
    \item Bei Entfernen der Kommentarsymbole // meldet der Java-Compiler einen Fehler.
    \item t.s hat den Wert null.
\end{itemize}


\section*{Anmerkungen und Ergänzungen}

Die Antworten lassen sich leicht durch Anwendung des Wissens finden, das bereits auf die vorhergehenden Fragen angewendet wurde:\\

\code{s} ist kein Klassenattribut, sondern ein Objektattribut (im Skript erklärt ab Seite 178).\\

\code{t.ok} hat den Wert false, weil Klassen- und Objektattribute vom Typ \code{boolean} automatisch mit dem Wert \code{false} initialisiert
werden.
Das gilt allerdings \underline{nicht} für \textbf{lokale Variablen} - diese werden nicht automatisch initialisiert; folglich meldet der Compiler
nach Entfernen der Kommentarsymbole auch einen Fehler~\footnote{
    Default-Werte für primitive Datentypen sowie Hinweis auf Initialisierung im Skript auf Seite 41.
}:

\blockquote[\footnote{
The Java™ Tutorials - Primitive Data Types (Default Values): \url{https://docs.oracle.com/javase/tutorial/java/nutsandbolts/datatypes.html}
}]{
    Local variables are slightly different; the compiler never assigns a default value to an uninitialized local variable.
    If you cannot initialize your local variable where it is declared, make sure to assign it a value before you attempt to use it.
    Accessing an uninitialized local variable will result in a compile-time error.

}



