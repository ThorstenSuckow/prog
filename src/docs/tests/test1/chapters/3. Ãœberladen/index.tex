\chapter{Überladen}\label{ueberladen}

\section*{Lösung}

\begin{itemize}
    \item Die Namen der Methoden sind gleich
    \item Der Rückgabe-Typ kann verschieden sein
\end{itemize}


\section*{Anmerkungen und Ergänzungen}

Im Skript wird das Überladen von Methoden ab Seite 166 behandelt.\\


Bei der Methodenüberladung wird die Parameterliste der Signatur einer Methode geändert (und wahlweise auch der Rückgabetyp).

In Java gehört zu der Methodensignatur der Methodenname sowie die formale Parameterliste\footnote {
    Java Language Specification - 8.4.2. Method Signature: \url{https://docs.oracle.com/javase/specs/jls/se21/html/jls-8.html#jls-8.4.2}
}.

Um eine Methode zu überladen, muss eine Methode erstellt werden, die den gleichen Namen wie die zu überladende Methode besitzt.
Die Parameterliste muss abgewandelt werden\footnote{The Java™ Tutorials - Defining Methods \url{https://docs.oracle.com/javase/tutorial/java/javaOO/methods.html}}.

In dem folgenden Beispiel sieht man leicht, dass bei der Methodenüberladung auch eine Änderung des Rückgabetyps Sinn machen kann:

\begin{lstlisting}[language=java]

class Foo {

    public int sum(int x, int y) {
        return x + y;
    }

    public double sum(double x, double y) {
        return x + y;
    }
}

\end{lstlisting}

Dass der Compiler trotz unterschiedlicher Parameterliste Schwierigkeiten haben kann, den ``richtigen`` Methodenaufruf zu finden,
behandelt das Skript auf Seite 167 (unten).\\

Das folgende Programm demonstriert dies:

\begin{lstlisting}[language=java]
class Foo {

    public double sum(double x, int y) {
        return x + y;
    }

    public double sum(int x, double y) {
        return x + y;
    }

    public static void main(String args[]) {

        Foo f = new Foo();

        System.out.println(f.sum(0, 0));

    }
}
\end{lstlisting}

Es wird der folgende Compiler-Fehler produziert:

\begin{lstlisting}[language=text]
error: reference to sum is ambiguous
        System.out.println(f.sum(0, 0));
                            ^
  both method sum(double,int) in Foo and method sum(int,double) in Foo match
\end{lstlisting}


Diesbzgl. könnte man dazu neigen, die Antwort ``Beim Aufruf der Methoden kann es zu Verwechslungen kommen.`` mit in
die Lösung einzubeziehen.
Allerdings muss zuerst der Begriff ``Aufruf`` im richtigen Kontext verstanden werden. Der Compiler stellt
fest, ob es zu Verwechslungen kommen kann: Ist das der Fall, unterbricht der Kompiliervorgang mit einer Fehlermeldung.
Dadurch wird ausgeschlossen, dass es während der Laufzeit - also beim ``Aufruf`` - zu Verwechslungen kommen kann.\\

Das Schlüsselwort \code{this} hat mit Methodenüberladung nichts zu tun.