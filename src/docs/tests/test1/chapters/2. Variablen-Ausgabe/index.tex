\chapter{Variablen-Ausgabe}

\section*{Lösung}

\begin{enumerate}
    \item a=2
    \item b=27
    \item c=5
    \item m(a)=25
    \item a=5
    \item b=4
    \item c=6
\end{enumerate}


\section*{Anmerkungen und Ergänzungen}

Klassen-und Objektattribute (sowie -Methoden) werden in dem Skript ab Seite 178 behandelt.

Besonderes Augenmerk ist bei der Analyse dem Modifizierer \code{static}\footnote{
    The Java™ Tutorials - Understanding Class Members: \url{https://docs.oracle.com/javase/tutorial/java/javaOO/classvars.html}
} zu widmen, der in der Aufgabe für
die Klassenattribute \code{a}, \code{b}, \code{c} und die Klassenmethode \code{m} verwendet wird.

In diesem Zusammenhang muss beachtet werden, dass der Parameter \code{c} der Methode \code{m} das Klassenattribut \code{c}
überdeckt\footnote{
    siehe hierzu Seite 213 im Skript
}. Um von \code{m} aus auf das Klassenattribut \code{c} zuzugreifen, wäre bei dem vorliegenden Code ein expliziter Zugriff
über den Klassennamen notwendig.\\

Folgendes Beispiel illustriert das Überdecken eines statischen Klassenattributs durch einen Methodenparameter:

\begin{lstlisting}[language=java]
class Foo {

    static int c = 42;

    static void m(int c) {
        System.out.println("Argument c: " + c);
        System.out.println("Klassenattribut c: " + Foo.c);

    }

    public static void main(String[] args) {
        Foo.m(11);
    }

}
\end{lstlisting}

Die Ausgabe lautet hier

\begin{lstlisting}[language=text]
Argument c: 11
Klassenattribut c: 42
\end{lstlisting}

Die Verwendung von \code{this.c} würde hingegen in der methode \code{m} zu einem Compiler-Fehler führen: \code{this} is ein
Objektattribut und funktioniert in einem statischen Kontext nicht:

\begin{lstlisting}[language=text]
non-static variable this cannot be referenced from a static context
\end{lstlisting}

Siehe hierzu auch Seite 181 im Skript.

