\chapter{Variablen-Ausgabe}

Die richtige Reihenfolge lautet:

\begin{enumerate}
    \item a=2
    \item b=27
    \item c=5
    \item m(a)=25
    \item a=5
    \item b=4
    \item c=6
\end{enumerate}


\section*{Anmerkungen und Ergänzungen}

Klassen-und Objetattribute (sowie -Methoden) werden in dem Skript ab Seite 178 behandelt.

Wichtig ist es, den Modifizierer \textit{static}\footnote{
    The Java™ Tutorials - Understanding Class Members: \url{https://docs.oracle.com/javase/tutorial/java/javaOO/classvars.html}
} zu beachten, der in der Aufgabe für die Klassenattribute ``a``, ``b``, ``c`` und die Klassenmethode ``m`` verwendet wird.

Ferner ist zu beachten, dass der Parameter ``c`` der Methode ``m`` das Klassenattribut ``c`` überdeckt~\footnote{siehe hierzu Seite 213 im Script}.
Um von ``m`` aus auf das Klassenattribut ``c`` zuzugreifen, wäre ein expliziter Zugriff über den Klassennamen notwendig.

Beispiel:

\begin{lstlisting}[language=java]
class Foo {

    static int c = 42;

    static void m(int c) {
        System.out.println("Argument c: " + c);
        System.out.println("Klassenattribut c: " + Foo.c);

    }

    public static void main(String[] args) {
        Foo.m(11);
        // Ausgabe:
        // Argument c: 11
        // Klassenattribut c: 42
    }

}
\end{lstlisting}

Die Verwendung von ``this.c`` würde hingegen in der methode ``m`` zu einem Compiler-Fehler führen: ``this`` is ein
Objekt-Attribut und funktioniert in einem statischen Kontext nicht:

\begin{lstlisting}[language=text]
non-static variable this cannot be referenced from a static context
\end{lstlisting}

Siehe hierzu auch Seite 181 im Script.

