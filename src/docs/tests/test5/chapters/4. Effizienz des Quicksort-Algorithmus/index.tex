\chapter{Effizienz des Quicksort-Algorithmus}\label{ch:quicksort}

\section*{Lösung}

\begin{itemize}
    \item von den gewählten Vergleichswerten
    \item von der Rekursionstiefe
\end{itemize}


\section*{Anmerkungen und Ergänzungen}

Die Vergleichswerte können durch die Implementierung gewählt werden, sie bedingen die Schrittfolgen.
Da die Schrittfolgen nicht gewählt werden können, fällt die Option als richtige Antwort raus.\\
Die Rekursionstiefe bedingt die Effizienz eines Algorithmus: Je tiefer diese ist, desto mehr Operationen werden durchgeführt.\\

Wichtige Maße für die Effizienz eines Algorithmus sind der benötigte Speicherplatz sowie die benötigte Rechenzeit zur Ausführung (vgl. \cite[2]{OW17a}, außerdem Skript (Teil 2) S. 157).
\\
Das Skript (Teil 2) stellt auf Seite 177 fest, dass der Aufwand für das Sortieren mit Quicksort wesentlich dadurch bestimmt wird, wie die Folgen in Teilfolgen aufgeteilt werden: So sind bspw. für Folgen der Länge $n$ Teilfolgen der Längen $1$ und $n-1$ möglich, wodurch ein Aufrufbaum \textit{entarten} kann und hohe Rekursionstiefe entsteht (vgl. \cite[177 f.]{GD18e})
\\

\textit{Ottmann und Widmayer} stellen fest:

\blockquote[{\cite[96]{OW17b}}]{
    Die im ungünstigsten Fall auszuführende Anzahl von Schlüsselvergleichen und Bewegungen
    hängt damit stark von der Anzahl der Aufteilungsschritte und damit von der Zahl der
    initiierten rekursiven Aufrufe ab.
}

Sie stellen zur Verbesserung der Effizienz die 3-Median-Strategie vor (vgl. \cite[102]{OW17b} sowie \cite[183]{GD18e}), die auch im Skript (Teil 2) auf Seite 177 zur Auswahl des Pivotelements gezeigt wird.