\chapter{Bubblesort}

\section*{Lösung}

\begin{itemize}
    \item Benachbarte Elemente werden vertauscht, wenn sie in der falschen Reihenfolge sind.
    \item Der Algorithmus ist stabil.
\end{itemize}



\section*{Anmerkungen und Ergänzungen}

Beide Aussagen finden sich im Skript (Teil 2) auf Seite 165, auf Seite 166 außerdem ein Beispiel, das das Vertauschen direkt benachbarter Elemente demonstriert.
Etwas formaler geben \textit{Ottmann und Widmayer} in \cite[89 ff.]{OW17b} eine Laufzeitanalyse an.
\\

Mit \textit{``Direkte Auswahl``} bezeichnet man auch \textbf{Selection-Sort}\footnote{ s. Skript (Teil 2) S. 159}.
\\

Bubblesort gehört zu den \textit{allgemeinen Sortierverfahren} und besitzt deshalb eine untere Schranke von $\Omega(n\ log\ n)$.
Aus diesem Grund werden auch kleinere unsortierte Mengen nicht in $O(n)$ sortiert\footnote{
s. Skript (Teil 2) S.179 ff., außerdem \cite[153 ff.]{OW17b}.
}, es findet außerdem keine sequentielle Suche nach dem kleinsten Element statt.

