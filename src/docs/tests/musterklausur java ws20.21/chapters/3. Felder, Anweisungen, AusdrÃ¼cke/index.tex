\chapter{Felder, Anweisungen und Ausdrücke}

\section{Lösungsvorschlag}

\begin{itemize}
    \item \code{3, 6, 2, 8, 5}
    \item \code{3, -12, -6, 32, 25}
    \item \code{3, -2, 8, 5, -6}
\end{itemize}

\section{Anmerkung und Ergänzungen}

Eckige Klammern sollten bei der Variablendeklaration {bzw.} bei der Typdeklaration von Feldern hinter dem Datentyp stehen, \textit{nicht} hinter dem Variablennamen\footnote{
    s.a. \url{https://docs.oracle.com/javase/tutorial/java/nutsandbolts/arrays.html} - abgerufen 9.2.2024}:

\begin{minted}[mathescape,
    numbersep=5pt,
    gobble=2,
    framesep=2mm]{java}
    // statt
    int numbers[]
    // besser
    int[] numbers
\end{minted}

\begin{itemize}
    \item Die Lesbarkeit wird verbessert: Die eckigen Klammern direkt hinter dem Typ weisen darauf hin, dass die Variable/der Parameter ein \textbf{Feld} von Werten des entsprechenden Typs ist.
    \item Bei folgendem Code ist nicht direkt ersichtlich, was gemeint ist:
    \begin{minted}[mathescape,
        numbersep=5pt,
        gobble=2,
        framesep=2mm]{java}
    int[] vector, matrix[];
    \end{minted}
    Die Schreibweise ist äquivalent zu
    \begin{minted}[mathescape,
        numbersep=5pt,
        gobble=2,
        framesep=2mm]{java}
    int vector[], matrix[][];
    // bzw.
    int[] vector;
    int[][] matrix;
    \end{minted}
    Die Sprachspezifikationen weisen daraufhin, dass diese Schreibweise nicht empfohlen wird:
    \blockquote[{``10.2. Array Variables``: \url{https://docs.oracle.com/javase/specs/jls/se21/html/jls-10.html#jls-10.2} - abgerufen 9.2.2024}]{
        We do not recommend ``mixed notation`` in array variable declarations, where bracket pairs appear on both the type and in declarators; nor in method declarations, where bracket pairs appear both before and after the formal parameter list.
    }
\end{itemize}\\

\noindent
Gegenüber folgender Schreibweise

\begin{minted}[mathescape,
    numbersep=5pt,
    gobble=2,
    framesep=2mm]{java}
    int[] numbers = new int[]{1, 2, 3, 4};
\end{minted}

ist folgende Schreibweise

\begin{minted}[mathescape,
    numbersep=5pt,
    gobble=2,
    framesep=2mm]{java}
    int[] numbers = {1, 2, 3, 4};
\end{minted}

\noindent
übersichtlicher, redundante Informationen sind weggefallen.\\
Beides ist erlaubt und führt am Ende zur Initialisierung des \code{int}-Arrays \code{[1, 2, 3, 4]}.\\
Eine Übergabe von \code{{1, 2, 3, 4}} als Argument an eine Methode, deren Signatur bspw.


\begin{minted}[mathescape,
    numbersep=5pt,
    gobble=2,
    framesep=2mm]{java}
    foo(int[])
\end{minted}

\noindent
entspricht, ist dagegen nicht möglich; hier muss \code{new int[]{1, 2, 3, 4}} verwendet werden.\\

\noindent
Bei mehrdimensionalen Arrays muss bei der Deklaration der lokalen Variable / des formalen Parameters immer die Anzahl der eckigen Klammern \code{[]} mit der Anzahl der Dimensionen des Arrays übereinstimmen.\\
Bei der Initialisierung eines mehrdimensionalen Arrays muss immer mindestens eine führende Dimension mit der Anzahl aufzunehmender Werte bestimmt werden:

\begin{minted}[mathescape,
    numbersep=5pt,
    gobble=2,
    framesep=2mm]{java}
    // funktioniert nicht:
    int[][] numbers = new int[][];
    int[][] numbers = new int[][4];

    // funktioniert:
    int[][] numbers = new int[4][];
\end{minted}

In Java sind mehrdimensionale Arrays \textbf{Arrays von Arrays} - sie müssen deshalb nicht rechteckig sein, wie das letzte Beispiel zeigt (vgl.~\cite[273 ff.]{Ull23}).