\chapter{Zeichenketten}

\section{Lösung}

\begin{minted}[mathescape,
    linenos,
    numbersep=5pt,
    gobble=2,
    fontsize=\small,
    frame=lines,
    framesep=2mm]{java}
    public static void main(String[] args) {
        int sum = 0;
        for (int i = 0; i < args.length; i++) {
            String arg = args[i];
            char c;
            int v;
            for (int j = 0; j < arg.length(); j++) {
                c = arg.charAt(j);
                v = c - '0';
                if (v >= 0 && v <= 9) {
                    sum += v;
                }
            }
        }
        System.out.println("Ergebnis: " + sum);
    }
\end{minted}\\

\section{Anmerkung und Ergänzungen}

Der Datentyp \code{char} ist in Java ein ganzzahliger Datentyp\footnote{
    ``4.2.1. Integral Types and Values``: \utl{https://docs.oracle.com/javase/specs/jls/se21/html/jls-4.html#jls-4.2.1} - abgerufen 10.2.2024
}, hat eine Länge von 16 Bit, ist vorzeichenlos (im Unterschied zu den anderen ganzzahligen Datentypen \code{byte}, \code{short}, \code{int}, \code{long}) und besitzt damit einen Wertebereich von $0 -- 65.535$.\\
Damit unterstützt \code{char} \textb{Unicode}-Zeichen im hexadezimalen Bereich von $0x0000$ bis $0xFFFF$\footnote{
    ``The Java Tutorials - Unicode``: \url{https://docs.oracle.com/javase/tutorial/i18n/text/unicode.html}
}.\\

\noindent
\code{char}-Variablen können als Ganzzahlwert ausgegeben werden, was dem Dezimalwert des Unicode-Zeichens entspricht\footnote{s. ``List of Unicode characters``: \url{https://en.wikipedia.org/wiki/List_of_Unicode_characters} - abgerufen 10.2.2024}:

\begin{minted}[mathescape,
    numbersep=5pt,
    gobble=2,
    framesep=2mm]{java}
    int toInt = 'c'; // entspricht dem Integer-Wert 99
\end{minted}

\noindent
Mittels \code{charAt(pos: int):char} der Klasse \code{String}\footnote{
    ``Class String``: \url{https://docs.oracle.com/en/java/javase/21/docs/api/java.base/java/lang/String.html#charAt(int)} - abgerufen 10.2.2024
} kann ein einzelner \textit{character} der \code{String}-Argumente dahingehend überprüft werden, ob dieser eine Ziffer repräsentiert.\\
Eine Ziffer bedeutet in diesem Fall ein \textit{character} aus der Liste $0..9$.\\
Dadurch, dass arithmetische Operationen auf dem Ganzzahl-Typ \code{char} möglich sind, ermittelt man für ein einzelnes Zeichen nun seine \textit{relative Position} (oder auch \textbf{Distanz}) zu dem Zeichen \code{'0'} - indem man diesen Wert einfach von dem zu vergleichenden Zeichen abzieht:

\begin{minted}[mathescape,
    numbersep=5pt,
    gobble=2,
    framesep=2mm]{java}
    int relC = 'c' - '0'; // 99 - 48 = 51
    int relA = 'a' - '0'; // 97 - 48 = 49
    int rel0 = '0' - '0'; // 48 - 48 = 0
    int rel7 = '7' - '0'; // 55 - 48 = 7
\end{minted}

Die Distanz liegt für die Ziffern $0..9$ dann in genau diesem Bereich, wodurch man den numerischen Wert des \code{chars} enthält.
Darüber läßt sich dann einfach die Summe berechnen.