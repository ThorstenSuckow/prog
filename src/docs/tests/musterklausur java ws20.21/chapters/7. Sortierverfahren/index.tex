\chapter{Sortierverfahren}

\section{Lösung}

\begin{itemize}
    \item Es gilt, dass jedes allgemeine Sortierverfahren mindestens $\Omega(n\ log\ n)$ Schlüsselvergleiche benötigt (vgl.~\cite[154]{OW17b}).\\
    Diese untere Schranke gilt sowohl für die \textit{elementaren Sortierverfahren}\footnote{s. ~\cite[81]{OW17b}}  (\textbf{Insertion}-,
    \textbf{Selection}- und \textbf{Bubble}-Sort), die im \textbf{worst-case} $O(n^2)$ Zeit benötigen, als auch für die
    Verfahren, die eine \textbf{divide and conquer}-Strategie implementieren, wie \textbf{Quicksort} ($O(n^2)$ worst-case)
    und \textbf{Merge-Sort} ($O(n\ log\ n)$ worst case).\\
    \item Bei sehr wenigen Datensätzen (mit wie in der Aufgabe angegeben $\leq 100$) ist die Wahl des Sortierverfahren
    hinsichtlich Effizienz im Sinne von Speicherplatzverbrauch als Laufzeit eher nebensächlich, wenn man davon ausgeht,
    dass das Sortieren auf einem der heutigen Technik entsprechenden Rechner mit einem der o.a. Verfahren durchgeführt wird.\\
    Qualitätskriterien wie Einfachheit der Implementierung und Verständlichkeit können hier eher ins Gewicht gefallen (vgl. \cite[5 f.]{GD18a}).
    \\
    Andere Kriterien, die die Eingabedaten betreffen, können jedoch die Auswahl des Sortierverfahrens beeinflussen: Sind die Daten überwiegend
    vorsortiert, kann bspw. \textbf{Insertion-Sort} verwendet werden, das bei vorsortierten Daten lineare Laufzeit erreichen kann
    \\
    Für die o.a. Sortierverfahren mit $O(n^2)$ im worst-case ergeben sich bei einer Eingabelänge von $100$ Datensätzen $\frac{n}{2} = ~5.000$ Operationen\footnote{
    Nachweise bspw. für Insertion Sort in \cite[87]{OW17b}, Selection Sort in \cite[172]{GD18e}, Bubble Sort in \cite{109}[Knu97b],
    Quicksort in \cite[97]{OW17b}
    }, für Merge-Sort $~600 - 700$ Operationen\footnote{
        Nachweis in \cite[116]{OW17b}
    }.
    \item Für größtenteils vorsortierte Daten eignet sich \code{Insertion Sort} besonders gut\footnote{\code{Ottmann und Widmayer} führen außerdem \textbf{Smoothsort} von Dijkstra an, dass $O(n)$ für eine vorsortierte Folge und $O(n\ log\ n)$ benötigt (vgl.~\cite[112]{OW17b})
\end{itemize}