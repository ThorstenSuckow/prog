\chapter{Klassen und Objekte}

\section{Lösung}

Die folgende Lösung enthält eine komplette Ausimplementierung der Klasse:

\begin{minted}[mathescape,
    linenos,
    numbersep=5pt,
    gobble=2,
    fontsize=\small,
    frame=lines,
    framesep=2mm]{java}
    public class Auto {

        private int kmStand;
        private double verbrauch;
        private double tankVolumen;
        private double kraftstoffVorrat;

        public Auto(int kmStand, double verbrauch, double tankVolumen, double kraftstoffVorrat) {
            this.kmStand = kmStand;
            this.verbrauch = verbrauch;
            this.tankVolumen = tankVolumen;
            this.kraftstoffVorrat = kraftstoffVorrat;
        }

        public String toString() {
            return "verbrauch: " + verbrauch + "; tankVolumen: " + tankVolumen +
                "; kmStand: " + kmStand + "; kraftstoffVorrat: " + kraftstoffVorrat;
        }

        public void fahren(int km) {

            if (km <= 0) {
                return;
            }

            double proKm = verbrauch / 100;

            int maxReichweite = (int) (kraftstoffVorrat / proKm);

            if (maxReichweite < km) {
                kmStand += maxReichweite;
                kraftstoffVorrat =0;
            } else {
                kmStand += km;
                kraftstoffVorrat -= proKm * km;
            }
        }

        public void tanken(double liter) {
            if (liter <= 0) {
                return;
            }

            if (liter + kraftstoffVorrat >= tankVolumen) {
                kraftstoffVorrat = tankVolumen;
            } else {
                kraftstoffVorrat += liter;
            }
        }

        public static void main (String[] args) {
            Auto gogoMobil = new Auto(0, 5.0, 50, 30);
            gogoMobil.fahren(300);
            gogoMobil.tanken(25);
        }
    }
\end{minted}\\

\section{Anmerkung und Ergänzungen}

\begin{itemize}
    \item Es ist darauf zu achten, die Aufgabenstellung hinsichtlich der Parameterbedingungen aufmerksam zu lesen.\\
    So ergibt sich bspw. für \code{tanken()}, dass die Methode nichts tut, wenn eine $0$ oder ein \textit{negativer} Wert übergeben wird - die Überprüfung der übergebenen Argumente resultiert in diesen Fällen in einer direkten Rückkehr aus der Methode.\\
    Gleiches gilt für \code{fahren()}, Werte für \code{int km} $\leq 0$ wirken sich nicht auf eine Änderung des Objekt-Zustands aus.
    \item Variablenüberdeckung ist o.k. wenn man bedenkt, dass auch das ASB-System diesbzgl. keinen checkstyle-Fehler produziert hat (s. \code{Flugzeug}-Aufgabe).
    \item \textbf{reelle Zahl} - \code{float} oder \code{double} benutzen?
    In Java ist \code{double} der Standarddatentyp für Fließkommazahlen\footnote{Konstanten wie $123.45$ werden im Quellcode automatisch als \textit{double} behandelt; sollen sie als \textit{float} behandelt werden, muss ein großes oder ein kleines ``f`` hintenangestellt werden: $1234.56f$ (vgl~\cite[124 f.]{Ull23})}.
    Ist der Datentyp nicht ausdrücklich beschrieben, sollte \code{double} ohne Einschränkung verwendet werden können.
\end{itemize}
