\chapter{Verkette Liste}



\section{Lösungsvorschlag}

\begin{minted}[mathescape,
    linenos,
    numbersep=5pt,
    gobble=2]{java}
    boolean istEnthalten(Object daten) {
        ListElement next = kopf;

        while (next != null) {
            if (next.getDaten().equals(daten)) {
                return true;
            }
            next = next.getNaechstes();
        }

        return false;
    }
\end{minted}

\section{Anmerkung und Ergänzungen}

Eine Liste ist eine \textbf{Sequenz} von Daten.
Auf \textbf{Sequenzen} ist eine Ordnung definiert - es gibt ein erstes, zweites, \ldots, letztes Element und zu jedem Element einen Vorgänger und Nachfolger (vgl.\cite[63]{GD18c}).\\
Sequenzen dürfen \textit{Duplikate} enthalten.\\

\\
In der Aufgabe ist die Implementierung einer einfach verketteten Liste (auch \textit{lineare Liste}, vgl. \cite[238]{Knu97a}, außerdem \cite[73]{GD18c}) gegeben:\\

\blockquote[{\cite[142]{SW11}}]{
A \textit{linked list} is a recursive data structure that is eitehr empty (\textit{null}) or a reference to a \textit{node} having a generic item and a reference to a linked list.
}\\

\noindent
Eine \textit{lineare Liste} hat ein Element, auf das kein anderes Element zeigt (\textit{head}, Angfang der Liste), und mindestens ein Element, welches auf kein weiteres Element zeigt (\textit{tail}, Ende der Liste)- ansonsten zeigt ein Element auf seinen Nachfolger (vgl. \cite[259]{CL22})\footnote{
Ebenda: ist eine Liste \textit{doppelt verkettet} (Element besitzt Zeiger auf Vorgänger\textit{prev}) und zeigt \textit{head} auf \textit{tail} und \textit{tail} auf \textit{head}, handelt es sich um eine \textit{zyklische Liste} (\textit{Ring}) (s.a. \cite[105]{GD18c})
}.

\noindent
Es gibt unterschiedliche Implementierungsmöglichkeiten für lineare Listen, die entsprechende Zugriffsreihenfolgen auf die in der Liste gespeicherten Daten ermöglichen, u.a. \textbf{Stack} und \textbf{Queue}.\\

\subsection*{Stack}
Ein \textbf{Stack}\footnote{
auch \textit{Keller} oder \textit{Stapel}
} arbeitet nach dem \textbf{LIFO}-Prinzip: Das Element, was als letztes (\textit{auf} dem Stack) hinzugefügt wurde, wird als erstes wieder entnommen.\\
Eingefügt wird am Ende des Stacks.\\

\subsection*{Queue}
Im Gegensatz zum Stack arbeitet eine \textbf{Queue}\footnote{
    auch \textit{(Warte)schlange}
} nach dem \textbf{FIFO}-Prinzip: Das Element, was als erstes hinzugefügt wurde, wird als erstes entnommen.\\
Wie beim Stack wird auch bei der Queue am Ende der Liste eingefügt.\\

\noindent
Die Kosten für Einfügen und Entfernen (``Entnahme``} sind für beide Listenarten konstant mit $O(1)$.\\
Suchoperationen benötigen im \code{best-case} $O(1)$, im \code{worst-case} $O(n)$. Im Durchschnitt muss man nur die Hälfte der Liste nach einem Element durchsuchen, bis es gefunden wird, was im \textbf{average-case} wieder zu einer linearen Laufzeit von $O(n)$ führt ($\frac{n}{2} = \frac{1}{2} * n \implies O(n)$).


\subsection*{Warum equals()?}
Die Klasse \code{java.lang.Object} als Elternklasse jeder anderen Klasse in Java\footnote{
    ``4.3.2. The Class Object``: \url{https://docs.oracle.com/javase/specs/jls/se8/html/jls-4.html#jls-4.3.2} - abgerufen 13.2.2024
} implementiert die Methode \code{equals(obj: Object):boolean} wie folgt:
\blockquote[{``equals``:  \url{https://docs.oracle.com/en/java/javase/21/docs/api/java.base/java/lang/Object.html#equals(java.lang.Object)} - abgerufen 13.2.2024}]{
    The equals method for class Object implements the most discriminating possible equivalence relation on objects; that is, for any non-null reference values x and y, this method returns true if and only if x and y refer to the same object (x == y has the value true). In other words, under the reference equality equivalence relation, each equivalence class only has a single element.
}. \\

Die Methode \code{equals()} sollte in Unterklassen überschrieben, \textit{nicht} überladen werden (vgl.~\cite[654]{Ull23}, weitere Hinweise auf die korrekte Implementierung der Methode ebenda, ausserdem bei \textit{Bloch} in \cite[37]{Blo18}).\\

\noindent
Da keine Angaben über den speziellen Typ von dem in der Liste gespeicherten Objekt gegeben ist, würde der Vergleich auf Identität spätestens bei Typen wie \code{String} dazu führen, dass Elemente nicht gefunden werden:

\begin{minted}[mathescape,
    linenos,
    numbersep=5pt,
    gobble=2]{java}
    boolean istEnthalten(Object daten) {
        ListElement next = kopf;

        while (next != null) {
            // ==, Vergleich auf Identität
            if (next.getDaten() == daten) {
                return true;
            }
            next = next.getNaechstes();
        }

        return false;
    }

    // liefert immer false zurück, auch wenn vorher über
    // liste.add(new String("foo")) ein entsprechendes Eement mit gleichem Inhalt
    // hinzugefügt wurde
    boolean enthalten = liste.istEnthalten("foo");
\end{minted}

In einer korrekten Implementierung von \code{istEnthalten} wird nun die \code{equals()}-Implementierung der Klasse \code{String} aufgerufen, die einen \textit{inhaltlichen} Vergleich vornimmt, und mit \begin{center}\code{new String("foo").equals("foo")}\end{center} auch \code{true} zurückliefert, im Gegensatz zu \code{new String("foo") == stringVariable}, wobei \code{stringVariable} vom Typ \code{String} ist und inhaltlich mit \code{"foo"} übereinstimmt.\\

\begin{tcolorbox}[title=String-Literale]
Eine Verwendung von \code{==} \textit{kann} bei dem Vergleich von \textbf{String-Literalen} \code{true} zurückliefern, was daran liegt, das String-Literale in einer Art Cache vorgehalten werden, damit die JVM nicht laufend neue String-Objekte erzeugen muss:\\

\blockquote[{``intern``: \url{https://docs.oracle.com/en/java/javase/21/docs/api/java.base/java/lang/String.html#intern()} - abgerufen 13.2.2024}]{
    All literal strings and string-valued constant expressions are interned.
 }\\

\noindent
Die Sprachspezifikation beinhaltet diesbzgl. weitere Informationen (s. ``3.10.5. String Literals``: \url{https://docs.oracle.com/javase/specs/jls/se21/html/jls-3.html#jls-3.10.5} - abgerufen 13.2.2024).
\end{tcolorbox}