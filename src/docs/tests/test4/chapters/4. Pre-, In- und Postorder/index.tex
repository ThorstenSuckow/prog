\chapter{Pre-, In- und Postorder}

\section*{Lösung}

\begin{itemize}
    \item \textbf{Preorder}: $35, 20,  10,  12,  24,  42,  40,  38,  80$
    \item \textbf{Inorder}: $10,  12,  20,  24,  35,  38,  40,  42,  80$
    \item \textbf{Postorder}: $12,  10,  24,  20,  38,  40,  80,  42,  35$
\end{itemize}


\section*{Anmerkungen und Ergänzungen}

Die Durchlaufordnungen werden im Skript (Teil 2) auf Seite 106 f. erklärt.
\\


\textit{Ottmann und Widmayer} führen für die Durchlaufordnung in Binärbäumen die Bezeichnungen \textbf{Hauptreihenfolge},
\textbf{Nebenreihenfolge} und \textbf{Symmetrische Reihenfolge} an (vgl.~\cite[272]{OW17e}):

\begin{itemize}
    \item \textbf{Hauptreihenfolge}: \textit{Preorder}
    \item \textbf{Symmetrische Reihenfolge}: \textit{Inorder}
    \item \textbf{Nebenreihenfolge} \textit{Postorder}
\end{itemize}

Die Präfixe ``Pre``, ``Post`` und ``In`` dürfen als Eselsbrücke dienen, denn sie beziehen sich auf die Reihenfolge, in der die Knoten betrachtet werden:

\begin{itemize}
    \item \textbf{Pre}: \textbf{Knoten zuerst}, dann linker Teilbaum, dann rechter Teilbaum (\textit{KLR}\footnote{L = Linker Teilbaum, R = Rechter Teilbaum, K = Knoten})
    \item \textbf{In}: erst linker Teilbaum, \textbf{danach Knoten}, dann rechter Teilbaum (\textit{LKR})
    \item \textbf{Post}:  erst linker Teilbaum, dann rechter Teilbaum, \textbf{zuletzt Knoten} (\textit{LRK})
\end{itemize}
