\chapter{Verkettete Listen und Arrayimplementierung}

\section*{Lösung}

\begin{itemize}
    \item Einfügeoperationen können bei der Arrayimplementierung einen höheren Aufwand erfordern als bei einer verketteten Implementierung.
\end{itemize}


\section*{Anmerkungen und Ergänzungen}


\subsection*{Löschoperationen der einfach verketteten Listenimplementierug arbeiten immer mit gleichem Aufwand wie Löschoperationen der Arrayimplementierung.}

Löschoperationen der Arrayimplementierung\footnote{
bei \textit{Ottmann und Widmayer} als ``Sequenziell gespeicherte lineare Liste`` bezeichnet (vgl. \cite[30 ff.]{OW17a})
} erfordern unter Umständen ein Verschieben nachfolgender Elemente um eine Position nach links in dem Array.
Wird {bspw.} das Element an Position $0$ gelöscht, hat diese Operation den Aufwand $O(1)$, wenn die Liste nur einen Eintrag enthält.
Besitzt hingegen die Liste eine Größe von $n$ Einträgen, müssen die Positionen der verbliebenen $n-1$ Einträge aktualisiert werden\footnote{
s. Skript (Teil 2) S. 89 ``Nachteile der Darstellung als Array``
}.
Wird hingegen das Element an der letzten Position gelöscht, ist kein Verschieben nachfolgender Elemente nötig, der Aufwand ist hier mit $O(1)$ konstant.
\\

Löschoperationen in einer einfach verketteten Liste besitzen i.d.R. eine mittlere Laufzeit von $O(n)$, da in der Liste zu dem \underline{Vorgängerelement} des zu löschenden Elements gelaufen werden muß, um dessen Zeiger zu aktualisieren, und zwar auf den Nachfolger des zu löschenden Elements\footnote{
Skript (Teil 2) S. 82. Auf Seite 83 ist eine alternative Realisierung angegeben, in der Löschoperationen in $O(1)$ möglich sind.
}.
Bei $n$ Einträgen in der Liste ist dann Löschen des letzten Elements mit einem Aufwand von $O(n)$ verbunden.


\subsection*{Einfügeoperationen können bei der Arrayimplementierung einen höheren Aufwand erfordern als bei einer verketteten Implementierung.}
Einfügeoperationen bei der Arrayimplementierung erfordern nicht nur ein Verschieben von Folgeelementen, sondern {ggfl.} auch eine Anpassung der Größe des Arrays\footnote{
    s. Skript (Teil 2) S. 89 ``Nachteile der Darstellung als Array``. Der Fall wird bei \textit{Ottmann und Widmayer} in \cite[32]{OW17a} nicht berücksichtigt: Ist die Liste bei der Einfügeoperation bereits voll, wird ein Fehler ausgegeben.
}.
Einfügeoperationen in verketteten Listen benötigen hingegen eine konstante Laufzeit\footnote{
    wenn das Element an der Zielposition erst ermittelt werden muss, ändert sich die Laufzeit zu $O(n)$.
}.

\subsection*{Die doppelt verkettete Liste und die Arrayimplementierung müssen die Positions- und Zeigerwerte von jedem Element nach jeder Einfüge- und Löschoperation aktualisieren.}
Bei der Einfügeoperation in einer doppelt verketteten Liste werden die Zeiger eines Vorgängerelements und des eingefügten Elements aktualisiert.
Der Rest der Liste bleibt unberührt\footnote{
das ist allerdings von der Implementierung abhängig. Sollte bspw. noch ein Element existieren, welches das Ende der Liste repräsentiert, muss dieses ggfl. noch mit aktualisiert werden (vgl. \cite[35 ff.]{OW17a}).
}.
Auch bei der Arrayimplementierung muss in dem Fall \underline{nicht jedes Element} aktualisiert werden - im wort case müssen $n-1$ Folgeelemente aktualisiert werden.


\subsection*{Das Einfügen am Beginn des Array benötigt immer Zeit $O(1)$.}
Wie bis hierher gesehen, bedingt das Einfügen eines Elements am Beginn eines Arrays eine Laufzeit von $O(n)$, falls $n-1$ Folgeelemnte verschoben werden müssen.
Muss das Array vergrößert werden, erhöht das auch die Laufzeit\footnote{
    mit einer neuen Größe von $m$ freien Plätzen entspricht das einer Laufzeit von $O(m)$. Siehe hierzu Skript (Teil 2) S. 89 ``Nachteile der Darstellung als Array``.
}.

\subsection*{Falls bereits $(n-1)$-Elemente im Array liegen, dann benötigt die Einfügeoperation am Schluss Zeit $O(n)$.}
Die Einfügeoperation im Array am Schluss des Arrays ist in konstanter Zeit $O(1)$ möglich, da keine Elemente verschoben werden müssen\footnote{
    bei der Fragestellung ist davon auszugehen, dass das Array nicht vergrößert werden muss.
}.

